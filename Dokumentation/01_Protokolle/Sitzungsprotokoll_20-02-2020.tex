\documentclass[
	ngerman,
	toc=listof, % Abbildungsverzeichnis sowie Tabellenverzeichnis in das Inhaltsverzeichnis aufnehmen
	toc=bibliography, % Literaturverzeichnis in das Inhaltsverzeichnis aufnehmen
	footnotes=multiple, % Trennen von direkt aufeinander folgenden Fußnoten
	parskip=half, % vertikalen Abstand zwischen Absätzen verwenden anstatt horizontale Einrückung von Folgeabsätzen
	numbers=noendperiod % Den letzten Punkt nach einer Nummerierung entfernen (nach DIN 5008)
]{scrartcl}
\pdfminorversion=5 % erlaubt das Einfügen von pdf-Dateien bis Version 1.7, ohne eine Fehlermeldung zu werfen (keine Garantie für fehlerfreies Einbetten!)

% Dokumenteninformationen ----------------------------------------------------
\newcommand{\titel}{Initial-Sitzung}
\newcommand{\untertitel}{Studienarbeit \semester}
\newcommand{\kompletterTitel}{\titel{} \\ \untertitel}
\newcommand{\datum}{20.02.2020}

\newcommand{\vorlagenOrdner}{../99_Vorlagen} % Falls im Unterordner ../ vorne hinzufügen

\newcommand{\betriebLogo}{\vorlagenOrdner/Bilder/logo}

% Konfiguration -------------------------------------------------------------
\newcommand{\autoren}{
    \author{
        Schmid, Mike\\
        \texttt{sgschwin@hsr.ch}
        \and
        Schlatter, Janik\\
        \texttt{jschlatt@hsr.ch}
    }
}

\newcommand{\betreuer}{
    Stettler Beat\\
    \scriptsize \texttt{\url{beat.stettler@hsr.ch}}
    \normalsize
}

\newcommand{\schmid}{
    Mike Schmid\\
    \url{mschmid@hsr.ch}
    \normalsize
}

\newcommand{\schlatter}{
    Janik Schlatter\\
    \scriptsize \url{jschlatt@hsr.ch}
    \normalsize
}

\newcommand{\autorenNamen}{
    M. Schmid, J. Schlatter
}

\newcommand{\semester}{FS-2020}
\newcommand{\betriebName}{\textsc{HSR} Hochschule für Technik Rapperswil} % Metadaten zu diesem Dokument (Autor usw.)
\input{\vorlagenOrdner/Konfiguration/Packages} % verwendete Packages
\input{\vorlagenOrdner/Konfiguration/Seitenstil_Protokoll} % Definitionen zum Aussehen der Seiten
\input{\vorlagenOrdner/Konfiguration/Befehle} % eigene allgemeine Befehle, die z.B. die Arbeit mit LaTeX erleichtern

\begin{document}
\pagenumbering{arabic}
\begin{center}
    \LARGE \textbf{\titel} \\[2ex]
    \large \datum \\[2ex]
\end{center}

% Inhalt ---------------------------------------------------------------------
\section*{Sitzungsteilnehmer}
\begin{tabularx}{0.9\linewidth}{Xl}
	\toprule
	Wer & Email \\
	\midrule
	Mike Schmid & \scriptsize \url{mschmid@hsr.ch} \\
	Janik Schlatter & \scriptsize \url{jschlatt@hsr.ch} \\
	Beat Stettler & \scriptsize \url{beat.stettler@hsr.ch} \\
	Urs Baumann & \scriptsize \url{urs.baumann@hsr.ch} \\
	\bottomrule
\end{tabularx}

\section*{Traktanden}
\begin{enumerate}
    \item Definition der Use-Cases
    \item Liste der abzugebenden Dokumenten per Meilenstein
    \item Grösse, Art und Definition des zu testenden Netzwerks
    \item Liste der zu bearbeitenden Protokollen
    \item Testkaskadierung (Ping -> Traffic) sinvoll/erwünscht?
    \item Ist ein privates GitHub Repository nötig?
    \item Art der Abgabe (Form, etc.)
\end{enumerate}

\section*{Beschlüsse}
\begin{itemize}
	\item Der haupt Use-Case ist, ein Tool zu entwickeln, bei dem ein User (Network Engineer) Tests spezifizieren kann, welche dann auf einem belebigen Netzwerk automatisiert ausgeführt werden. Hierbei sollte auf eine gute Abstraktion geachtet werden.
	\item In erster Linie soll eine Definitionssprache für Tests ausgearbeitet werden, die als Abstraktionslayer zwischen der Testautomationssoftware und der Hardware dient.
	\item Die Dokumente und die Abgabe werden gemäss der von der HSR zur Verfügung gestellten Vorlage vorgenommen.
	\item Die Grösse, Art und Definition des zu testenden Netzwerks soll keine Rolle spielen. 
	\item Die Netzwerkprotokolle sind im Kontext der Arbeit nebensächlich. Die Autoren können selber bestimmen, welche Protokolle in welcher Reihenfolge abgearbeitet werden.
	\item Testkaskadierung sollte, wo nötig, eingesetzt werden.
	\item Das GitHub Repository wird als öffentliches Repository mit einer OpenSource-Lizenz erstellt.
	\item Es sollte eine History (Logs) erstellt werden, bei welcher man über die letzten Tage alle Testdurchläufe mit Resultaten ansehen kann.
\end{itemize}

\section*{Offene Punkte}
\begin{tabularx}{0.9\linewidth}{Xll}
	\toprule
	Was & Verantwortlicher & Datum \\
	\midrule
	GitHub Repository gemäss Beschlüssen anpassen & Mike Schmid & 25.02.2020 \\
	Projektplanung erstellen & Janik Schlatter, Mike Schmid & 27.02.2020 \\
	Mit Analyse der zu verwendenden Technologien beginnen & Janik Schlatter, Mike Schmid & 27.02.2020 \\
	\bottomrule
\end{tabularx}

\end{document}