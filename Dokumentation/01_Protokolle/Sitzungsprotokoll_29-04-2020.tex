\documentclass[
	ngerman,
	toc=listof, % Abbildungsverzeichnis sowie Tabellenverzeichnis in das Inhaltsverzeichnis aufnehmen
	toc=bibliography, % Literaturverzeichnis in das Inhaltsverzeichnis aufnehmen
	footnotes=multiple, % Trennen von direkt aufeinander folgenden Fußnoten
	parskip=half, % vertikalen Abstand zwischen Absätzen verwenden anstatt horizontale Einrückung von Folgeabsätzen
	numbers=noendperiod % Den letzten Punkt nach einer Nummerierung entfernen (nach DIN 5008)
]{scrartcl}
\pdfminorversion=5 % erlaubt das Einfügen von pdf-Dateien bis Version 1.7, ohne eine Fehlermeldung zu werfen (keine Garantie für fehlerfreies Einbetten!)

% Dokumenteninformationen ----------------------------------------------------
\newcommand{\titel}{Sitzung Umsetzung 4}
\newcommand{\untertitel}{Studienarbeit \semester}
\newcommand{\kompletterTitel}{\titel{} \\ \untertitel}
\newcommand{\datum}{29.04.2020}

\newcommand{\vorlagenOrdner}{../99_Vorlagen} % Falls im Unterordner ../ vorne hinzufügen

\newcommand{\betriebLogo}{\vorlagenOrdner/Bilder/logo}

% Konfiguration -------------------------------------------------------------
\newcommand{\autoren}{
    \author{
        Schmid, Mike\\
        \texttt{sgschwin@hsr.ch}
        \and
        Schlatter, Janik\\
        \texttt{jschlatt@hsr.ch}
    }
}

\newcommand{\betreuer}{
    Stettler Beat\\
    \scriptsize \texttt{\url{beat.stettler@hsr.ch}}
    \normalsize
}

\newcommand{\schmid}{
    Mike Schmid\\
    \url{mschmid@hsr.ch}
    \normalsize
}

\newcommand{\schlatter}{
    Janik Schlatter\\
    \scriptsize \url{jschlatt@hsr.ch}
    \normalsize
}

\newcommand{\autorenNamen}{
    M. Schmid, J. Schlatter
}

\newcommand{\semester}{FS-2020}
\newcommand{\betriebName}{\textsc{HSR} Hochschule für Technik Rapperswil} % Metadaten zu diesem Dokument (Autor usw.)
\input{\vorlagenOrdner/Konfiguration/Packages} % verwendete Packages
\input{\vorlagenOrdner/Konfiguration/Seitenstil_Protokoll} % Definitionen zum Aussehen der Seiten
\input{\vorlagenOrdner/Konfiguration/Befehle} % eigene allgemeine Befehle, die z.B. die Arbeit mit LaTeX erleichtern

\begin{document}
\pagenumbering{arabic}
\begin{center}
    \LARGE \textbf{\titel} \\[2ex]
    \large \datum \\[2ex]
\end{center}

% Inhalt ---------------------------------------------------------------------
\section*{Sitzungsteilnehmer}
\begin{tabularx}{0.9\linewidth}{Xl}
	\toprule
	Wer & Email \\
	\midrule
	Mike Schmid & \scriptsize \url{mschmid@hsr.ch} \\
	Janik Schlatter & \scriptsize \url{jschlatt@hsr.ch} \\
	Beat Stettler & \scriptsize \url{beat.stettler@hsr.ch} \\
	Urs Baumann & \scriptsize \url{urs.baumann@hsr.ch} \\
	\bottomrule
\end{tabularx}

\section*{Traktanden}
\begin{enumerate}
	\item Besprechung Zwischenstand der Implementierung
	\item Besprechung Feature Freeze Verschiebung
\end{enumerate}

\section*{Beschlüsse}
	\begin{itemize}
		\item Der Feature Freeze wird um fünf Tage verschoben.
		\item Die aktuelle Dokumentation und der aktuelle Code müssen auf das nächste Meeting abgegeben werden.
	\end{itemize}

\section*{Offene Punkte}
\begin{tabularx}{0.9\linewidth}{Xll}
	\toprule
	Was & Verantwortlicher & Datum \\
	\midrule
	 Implementierung von zusätzlichen Tests & Mike Schmid & 06.05.2020 \\
	 Aktualisierung der Dokumentation & Mike Schmid, Janik Schlatter & 06.05.2020 \\
	 Logger Implementieren & Janik Schlatter & 06.05.2020 \\
	\bottomrule
\end{tabularx}

\end{document}