\documentclass[
	ngerman,
	toc=listof, % Abbildungsverzeichnis sowie Tabellenverzeichnis in das Inhaltsverzeichnis aufnehmen
	toc=bibliography, % Literaturverzeichnis in das Inhaltsverzeichnis aufnehmen
	footnotes=multiple, % Trennen von direkt aufeinander folgenden Fußnoten
	parskip=half, % vertikalen Abstand zwischen Absätzen verwenden anstatt horizontale Einrückung von Folgeabsätzen
	numbers=noendperiod % Den letzten Punkt nach einer Nummerierung entfernen (nach DIN 5008)
]{scrartcl}
\pdfminorversion=5 % erlaubt das Einfügen von pdf-Dateien bis Version 1.7, ohne eine Fehlermeldung zu werfen (keine Garantie für fehlerfreies Einbetten!)

% Dokumenteninformationen ----------------------------------------------------
\newcommand{\titel}{Projektplan}
\newcommand{\untertitel}{Studienarbeit \semester}
\newcommand{\kompletterTitel}{\titel{} \\ \untertitel}
\newcommand{\datum}{\today}

\newcommand{\vorlagenOrdner}{../../99_Vorlagen} % Falls im Unterordner ../ vorne hinzufügen

\newcommand{\betriebLogo}{\vorlagenOrdner/Bilder/logo}

% Konfiguration -------------------------------------------------------------
\newcommand{\autoren}{
    \author{
        Schmid, Mike\\
        \texttt{sgschwin@hsr.ch}
        \and
        Schlatter, Janik\\
        \texttt{jschlatt@hsr.ch}
    }
}

\newcommand{\betreuer}{
    Stettler Beat\\
    \scriptsize \texttt{\url{beat.stettler@hsr.ch}}
    \normalsize
}

\newcommand{\schmid}{
    Mike Schmid\\
    \url{mschmid@hsr.ch}
    \normalsize
}

\newcommand{\schlatter}{
    Janik Schlatter\\
    \scriptsize \url{jschlatt@hsr.ch}
    \normalsize
}

\newcommand{\autorenNamen}{
    M. Schmid, J. Schlatter
}

\newcommand{\semester}{FS-2020}
\newcommand{\betriebName}{\textsc{HSR} Hochschule für Technik Rapperswil} % Metadaten zu diesem Dokument (Autor usw.)
\input{\vorlagenOrdner/Konfiguration/Packages} % verwendete Packages
\input{\vorlagenOrdner/Konfiguration/Seitenstil_Bericht} % Definitionen zum Aussehen der Seiten
\input{\vorlagenOrdner/Konfiguration/Befehle} % eigene allgemeine Befehle, die z.B. die Arbeit mit LaTeX erleichtern

\begin{document}

% Deckblatt ------------------------------------------------------------------
\phantomsection
\thispagestyle{plain}
\pdfbookmark[1]{Deckblatt}{deckblatt}
\begin{titlepage}
    \begin{center}
        \includegraphics[scale=1.5]{\betriebLogo}\\[10ex]

        \rule{\linewidth}{0.5mm}\\[2ex]
        {\huge \bfseries  \titel }\\[2ex]
        {\LARGE \untertitel }\\[2ex]
        {\large \datum}\\
        \rule{\linewidth}{0.5mm}\\[10ex]

        \begin{minipage}[t]{0.4\textwidth}
            \begin{flushleft} 
                \large \emph{Autoren:}\\
                    \large Mike \textsc{Schmid}\\
                    \scriptsize \texttt{mike.schmid@hsr.ch}\\[1ex]
                    \large Janik \textsc{Schlatter}\\
                    \scriptsize \texttt{janik.schlatter@hsr.ch}\\[1ex]
            \end{flushleft}
            \end{minipage}
            ~
            \begin{minipage}[t]{0.4\textwidth}
            \begin{flushright} 
                \large \emph{Supervisor:} \\
                Prof. Stettler \textsc{Beat}\\
                \scriptsize \texttt{beat.stettler@hsr.ch}\\[1ex]
            \end{flushright}
        \end{minipage}\\[40ex]

        \small
        \noindent
        Dieses Werk einschließlich seiner Teile ist \textbf{urheberrechtlich geschützt}.
        Jede Verwertung außerhalb der engen Grenzen des Urheberrechtgesetzes ist ohne
        Zustimmung des Autors unzulässig und strafbar. Das gilt insbesondere für
        Vervielfältigungen, Übersetzungen, Mikroverfilmungen sowie die Einspeicherung
        und Verarbeitung in elektronischen Systemen.

    \end{center}
\end{titlepage}
\cleardoublepage

% Preface --------------------------------------------------------------------
\pagenumbering{Roman}

% Zweck
\section*{Zweck}
Dieses Dokument beschreibt den Projektplan und liefert eine Übersicht über das Projekt Network Unit Testing System, dessen Planung und Organisation, sowie über weitere Bereiche des Projektaufbaus. Der Projektplan dient als Grundlage und Referenz für nachfolgende Projektdokumente

% Änderungsgeschichte
\section*{Änderungsgeschichte}
\begin{tabularx}{0.9\textwidth}{llXl}
	\toprule
	Datum & Version & Änderung & Autor \\
	\midrule
	20.02.2018 & 1.0 & Initial Setup & Janik Schlatter \\
	\bottomrule
\end{tabularx}
\cleardoublepage

% Inhaltsverzeichnis
\phantomsection
\pdfbookmark[1]{Inhaltsverzeichnis}{inhalt}
\tableofcontents
\cleardoublepage

\pagenumbering{arabic}
% Jede Überschrift 1 auf neuer Seite
\let\stdsection\section
\renewcommand\section{\clearpage\stdsection}

% Inhalt ---------------------------------------------------------------------
\section{Einführung}

	\subsection{Sprache}
		Wir haben uns entschieden für die allgemeine Projektsprache (Dokumentation, Use Cases, etc.) Deutsch zu wählen.
		Nur für den Code und die Versionskontrolle wird English als Sprache verwendet.

	\subsection{Referenzen}
		Alle Dokumente werden auf dem Git abgelegt.

	\begin{tabularx}{\textwidth}{lX}
		Git Repository & \url{https://github.com/EkoGuandor229/Network-Unit-Testing.git} \\
	\end{tabularx}

	\subsection{Vorarbeit NUTS}
		Es existiert bereits eine Vorarbeit, zu diesem Thema. BLABLABLA

\section{Projektübersicht}

	\subsection{Projektübersicht}

	\subsection{Zweck und Ziel}

	\subsection{Lieferumfang}
		Im Rahmen der Studienarbeit wird folgendes erstellt:
		\begin{itemize}
			\item Ein Framework mit blablabla
		\end{itemize}
	\subsection{Annahmen und Einschränkungen}

\section{Projektorganisation}

	\subsection{Projektmitglieder}
		\begin{tabularx}{0.9\textwidth}{lX}
			\toprule
			Name & Email \\
			\midrule
			Janik Schlatter & \url{jschlatt@hsr.ch} \\
			Mike Schmid & \url{mschmid@hsr.ch} \\
			\bottomrule
		\end{tabularx}

	\subsection{Externe Schnittstellen}
		\begin{tabularx}{0.9\textwidth}{lXl}
			\toprule
			Name & Email & Zuständikeit \\
			\midrule
			Beat Stettler & \url{beat.stettler@hsr.ch} & Betreuer \\
			Urs Baumann & \url{urs.baumgartner@hsr.ch} & Betreuer \\
			\bottomrule
		\end{tabularx}

\section{Management Abläufe}

	\subsection{Kostenvoranschlag}
		Das Projekt wurde am 20.02.2020 gestartet und wird voraussichtlich am 28.05.2020 enden.
		Das heisst, es stehen 14 Wochen Zeit zur Verfügung während dem Semester. 
		Jedes Projektmitglied arbeitet insgesamt 240 Stunden an dem Projekt, sprich 16 Stunden pro Woche pro Projektmitglied. 
		\begin{tabularx}{\textwidth}{Xl}
			\midrule
			Projektdauer & 14 Wochen \\
			\midrule
			Anzahl Projektmitglieder & 2 \\
			\midrule
			Arbeitsstunden pro Woche und Person & 16 \\
			\midrule
			Arbeitsstunden insgesamt & 480 \\
			\midrule
			Projektstart & 20.02.2020 \\
			\midrule
			Projektende & 28.05.2020 \\
			\midrule
		\end{tabularx}

	\subsection{Zeitliche Planung}
		Die 14 Wochen des Projekts werden in fünf Phasen unterteilt: Initialisierung, Analyse, Design, Realisierung und Abschluss.
		\begin{center}
			\label{ZeitplanOverview}
			\includegraphics[scale=0.8]{\vorlagenOrdner/Bilder/Projektphasen}
		\end{center}

	\subsection{Phasen/Iterationen}
	\minisec{Phasen}
		Wir halten uns an die folgenden 5 Phasen: \\
		\begin{tabularx}{\textwidth}{lll}
			\toprule
			Farbe* & Bezeichnung & Zeitrahmen \\
			\midrule
			\textcolor{green}{Grün} & Initialisierung & 1 Woche \\
			\textcolor{orange}{Orange} & Analyse & 3 Wochen \\
			\textcolor{red}{Rot} & Design & 3 Wochen \\
			\textcolor{blue}{Blau} & Realisierung & 5 Wochen \\
			\textcolor{yellow}{Gelb} & Abschluss & 2 Wochen \\
			\bottomrule
		\end{tabularx}
	
		\minisec{Iterationen}
		Die Iterationen werden wöchentlich gemacht. Da wir auch ein Mal wöchentlich das Meeting haben passt das gut aufeinander.


	\subsection{Meilensteine}
		\begin{tabularx}{0.9\linewidth}{lllX}
			\toprule
			Nr & Bezeichnung & Termin & Beschreibung \\
			\midrule
			M0 & Name & Datum & Bla \\
			\bottomrule
		\end{tabularx}

	\subsection{Besprechungen/Protokolle}
		Es wurden zwei Termine vereinbart, an welchen sich die Projektmitglieder treffen. 
		Bei beiden Terminen stehen jeweils mindestens 6 Lektionen zur Verfügung.

		\begin{tabularx}{0.9 \linewidth}{llX}
			\toprule
			Nr & Wann & Beschreibung \\
			\midrule
			1 & Dienstag 10:00 - 17:00 & Gemeinsame Arbeit der Projektmitglieder \\
			2 & Donnerstag 08:00 - 17:00 & Gemeinsame Arbeit der Projektmitglieder \\
			3 & Donnerstag 14:00 - 15:00 & Besprechung mit Projektbetreuern \\
			\bottomrule
		\end{tabularx}

\section{Risikomanagement}

	\subsection{Risiken}
		Eine Riskoanalyse mit gewichtetem Schaden und Informationen zur Vorbeugung ist auf der Ablage zu
		finden (siehe Dokument TechnischeRisiken.xlsx)

	\subsection{Umgang mit Risiken}
		Um Probleme gerade während der Init/Analyse Phase möglichst früh zu erkennen, 
		arbeiten wir wöchentlich zwei Tage nebeneinander, um uns über mögliche Probleme auszutauschen. 
		Desweiteren suchen wir auch den Kontakt zum Betreuer sobald Unklarheiten im Team herrschen. 

\section{Infrastruktur}
	Alle Arbeiten zum Projekt werden von den Projektmitgliedern auf Ihrem jeweiligen Laptop verrichtet.
	(Evt erhalten wir noch ein physikalische Netzwerk um zu testen?!?!?!)

	\subsection{Übersicht der Tools}
		Für die Umsetzung des Projektes werden folgende Tools verwendet: \\[2ex]
		\begin{tabularx}{0.9\linewidth}{lX}
			\toprule
			Bezeichung & Beschreibung \\
			\midrule
			Git & Versionsverwaltung \\
			PyCharm & Entwicklungsumgebung \\
			\bottomrule
		\end{tabularx}

\section{Qualitätsmassnahmen}
	\subsection{Allgemein}

	\subsection{Testing}

	\subsection{Besprechungen}

	\subsection{Versionskontrolle}
		Sämtliche Dokumente werden in einem Git Repository abgelegt. Damit wird, dank der Versionskontrolle, 
		jede Änderung nachvollziehbar sein und es können auf sämtlichen alten Versionen zugegriffen werden.

	\subsection{Dokumente}

	\subsection{Code-Qualität}

\end{document}