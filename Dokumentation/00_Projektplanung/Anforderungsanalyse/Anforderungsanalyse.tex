\documentclass[
	ngerman,
	toc=listof, % Abbildungsverzeichnis sowie Tabellenverzeichnis in das Inhaltsverzeichnis aufnehmen
	toc=bibliography, % Literaturverzeichnis in das Inhaltsverzeichnis aufnehmen
	footnotes=multiple, % Trennen von direkt aufeinander folgenden Fußnoten
	parskip=half, % vertikalen Abstand zwischen Absätzen verwenden anstatt horizontale Einrückung von Folgeabsätzen
	numbers=noendperiod % Den letzten Punkt nach einer Nummerierung entfernen (nach DIN 5008)
]{scrartcl}
\pdfminorversion=5 % erlaubt das Einfügen von pdf-Dateien bis Version 1.7, ohne eine Fehlermeldung zu werfen (keine Garantie für fehlerfreies Einbetten!)

% Dokumenteninformationen ----------------------------------------------------
\newcommand{\titel}{Anforderungsanalyse}
\newcommand{\untertitel}{Studienarbeit \semester}
\newcommand{\kompletterTitel}{\titel{} \\ \untertitel}
\newcommand{\datum}{\today}

\newcommand{\vorlagenOrdner}{../../99_Vorlagen} % Falls im Unterordner ../ vorne hinzufügen

\newcommand{\betriebLogo}{\vorlagenOrdner/Bilder/logo}

% Konfiguration -------------------------------------------------------------
\newcommand{\autoren}{
    \author{
        Schmid, Mike\\
        \texttt{sgschwin@hsr.ch}
        \and
        Schlatter, Janik\\
        \texttt{jschlatt@hsr.ch}
    }
}

\newcommand{\betreuer}{
    Stettler Beat\\
    \scriptsize \texttt{\url{beat.stettler@hsr.ch}}
    \normalsize
}

\newcommand{\schmid}{
    Mike Schmid\\
    \url{mschmid@hsr.ch}
    \normalsize
}

\newcommand{\schlatter}{
    Janik Schlatter\\
    \scriptsize \url{jschlatt@hsr.ch}
    \normalsize
}

\newcommand{\autorenNamen}{
    M. Schmid, J. Schlatter
}

\newcommand{\semester}{FS-2020}
\newcommand{\betriebName}{\textsc{HSR} Hochschule für Technik Rapperswil} % Metadaten zu diesem Dokument (Autor usw.)
\input{\vorlagenOrdner/Konfiguration/Packages} % verwendete Packages
\input{\vorlagenOrdner/Konfiguration/Seitenstil_Bericht} % Definitionen zum Aussehen der Seiten
\input{\vorlagenOrdner/Konfiguration/Befehle} % eigene allgemeine Befehle, die z.B. die Arbeit mit LaTeX erleichtern

\begin{document}

% Deckblatt ------------------------------------------------------------------
\phantomsection
\thispagestyle{plain}
\pdfbookmark[1]{Deckblatt}{deckblatt}
\begin{titlepage}
    \begin{center}
        \includegraphics[scale=1.5]{\betriebLogo}\\[10ex]

        \rule{\linewidth}{0.5mm}\\[2ex]
        {\huge \bfseries  \titel }\\[2ex]
        {\LARGE \untertitel }\\[2ex]
        {\large \datum}\\
        \rule{\linewidth}{0.5mm}\\[10ex]

        \begin{minipage}[t]{0.4\textwidth}
            \begin{flushleft} 
                \large \emph{Autoren:}\\
                    \large Mike \textsc{Schmid}\\
                    \scriptsize \texttt{mike.schmid@hsr.ch}\\[1ex]
                    \large Janik \textsc{Schlatter}\\
                    \scriptsize \texttt{janik.schlatter@hsr.ch}\\[1ex]
            \end{flushleft}
            \end{minipage}
            ~
            \begin{minipage}[t]{0.4\textwidth}
            \begin{flushright} 
                \large \emph{Supervisor:} \\
                Prof. Stettler \textsc{Beat}\\
                \scriptsize \texttt{beat.stettler@hsr.ch}\\[1ex]
            \end{flushright}
        \end{minipage}\\[40ex]

        \small
        \noindent
        Dieses Werk einschließlich seiner Teile ist \textbf{urheberrechtlich geschützt}.
        Jede Verwertung außerhalb der engen Grenzen des Urheberrechtgesetzes ist ohne
        Zustimmung des Autors unzulässig und strafbar. Das gilt insbesondere für
        Vervielfältigungen, Übersetzungen, Mikroverfilmungen sowie die Einspeicherung
        und Verarbeitung in elektronischen Systemen.

    \end{center}
\end{titlepage}
\cleardoublepage

% Preface --------------------------------------------------------------------
\pagenumbering{Roman}

% Änderungsgeschichte
\section*{Änderungsgeschichte}
\begin{tabularx}{\textwidth}{llXl}
	\toprule
	Datum & Version & Änderung & Autor \\
	\midrule
	27.02.2020 & 1.0 & Initial Setup & Janik Schlatter \\
	\bottomrule
\end{tabularx}
\cleardoublepage

% Inhaltsverzeichnis
\phantomsection
\pdfbookmark[1]{Inhaltsverzeichnis}{inhalt}
\tableofcontents
\cleardoublepage

\pagenumbering{arabic}
% Jede Überschrift 1 auf neuer Seite
\let\stdsection\section
\renewcommand\section{\clearpage\stdsection}

% Inhalt ---------------------------------------------------------------------
\section{Allgemeine Beschreibung}

\section{Use Cases}
	\subsection{Personas}
		\begin{tabularx}{\textwidth}{lXX}
			\toprule
			Person & Beschreibung & Technisches Wissen \\
			\midrule
			Net-Admin & Der Net-Administrator ist der Chef über das ganze Netzwerk und trägt die ganze Verantwortung darüber. & Der Net-Administrator hat grundlegendes Wissen über Python.\\
			\midrule
			Net-Engineer & Der Net-Engineer ist die Person, welche das Netwerk aufgesetzt hat und es am laufen hält. & Der Net-Engineer hat wissen über Python und sollte im Stande sein, das Programm in Betrieb zu nehmen. \\
			\midrule
			Net-Techniker & Der Net-Techniker unterstützt den Net-Engineer bei der Wartung und erledigt den Support. & Der Net-Techniker hat kein wissen über Python. Er ist nur im Stande das Programm zu benutzen, aber nicht aufzusetzen. \\
			\bottomrule
		\end{tabularx}

	\subsection{Use Cases Brief}

	\subsection{Use Case Diagramm}

\section{Nichtfunktionale Anforderungen}
	In diesem Kapitel behandeln wir die nichtfunktionalen Anforderungen an das Projekt.
	Wir behandeln Aspekte und Anforderungen aus den Bereichen Performance, Qualität und Sicherheit.
	
	\subsection{Performance}

	\subsection{Qualität}
		\subsubsection{Funktionalität}
		Netzwerkdevices können von vielen unterschiedlichen Herstellern kommen. 
		Diese Hersteller verwenden unterschiedliche Syntax und Ausgabeformate. 
		Um die Funktionalität bestmöglich sicherzustellen, wird die Herstellerunterstützung vorerst stark eingeschränkt. 
		Vorgesehen sind vorerst Cisco Netzwerkdevices und Linux Hosts. \textcolor{red}{TODO Update}

		\subsubsection{Zuverlässikeit}
		Tests sind wichtig und nützlich, jedoch nicht businesskritisch bei einem möglichen
		Ausfall. Es muss vor allem darauf geachtet werden, dass bei ssh Verbindungen ein
		sauberes Exception Handling implementiert wird, falls beim Verbindungsaufbau oder
		bei abgesetzten Kommandos etwas schief geht. Es müssen für gewisse Tests auch
		Timeouts eingeplant werden, damit das Programm nicht unendlich lange blockieren
		kann.\textcolor{red}{TODO Update}

		\subsubsection{Benutzbarkeit}
		Wir möchten ein schmales User Interface auf Konsolen Ebene bieten. Der Anwender
		muss über Kenntnisse auf der Linux Shell verfügen. Mittels eingebauter Hilfe soll es
		versierten Benutzern möglich sein, die Software zu verwenden. \textcolor{red}{TODO Update}

		\subsubsection{Effizienz}
		Die Software wird lokal betrieben. Die einzelnen Unit Tests laufen seriell ab. Die gesamt
		benötigte Ausführungszeit ist also die Summe aller Ausführungszeiten einzelner Tests. \textcolor{red}{TODO Update}

		\subsubsection{Wartbarkeit}
		Eigene Test Cases sollen mit den notwendigen Kenntnissen selbst ergänzt werden können. Command-Mapping und Outputs müssen bekannt sein, dann ist eine Erweiterung
		des Funktionsumfangs der Test Cases denkbar. \textcolor{red}{TODO Update}

		\subsubsection{Übertragbarkeit}
		Die Übertragbarkeit auf andere Plattformen oder Hersteller ist schwierig und vorerst
		nicht vorgesehen. \textcolor{red}{TODO Update}

	\subsection{Sicherheit}
	Um auf Devices verbinden zu können, muss man sich auf diesen authentifizieren. Administrator Zugänge müssen deshalb best möglichst geschützt sein. Die Übertragung
	muss verschlüsselt sein (SSH) und die Passwörter dürfen, wenn überhaupt, nur mittels
	sicherem Hashverfahren abgelegt werden. \textcolor{red}{TODO Update}

\section{Weitere Anforderungen}
	\subsection{Schnittstellen}
		NUTS hat verschiedene interne Schnittstellen, welche hier aufgezeigt werden: \\
		\begin{tabularx}{\textwidth}{lX}
			\toprule
			Schnittstelle & Beschriebung \\
			\midrule
			Benutzerschnittstelle & \textcolor{red}{TODO} \\
			Netzwerkschnittstelle & \textcolor{red}{TODO} \\
			\bottomrule
		\end{tabularx}

	\subsection {Randbedingungen}


\end{document}