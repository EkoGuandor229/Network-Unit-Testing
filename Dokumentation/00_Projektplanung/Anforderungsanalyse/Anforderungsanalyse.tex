\documentclass[
	ngerman,
	toc=listof, % Abbildungsverzeichnis sowie Tabellenverzeichnis in das Inhaltsverzeichnis aufnehmen
	toc=bibliography, % Literaturverzeichnis in das Inhaltsverzeichnis aufnehmen
	footnotes=multiple, % Trennen von direkt aufeinander folgenden Fußnoten
	parskip=half, % vertikalen Abstand zwischen Absätzen verwenden anstatt horizontale Einrückung von Folgeabsätzen
	numbers=noendperiod % Den letzten Punkt nach einer Nummerierung entfernen (nach DIN 5008)
]{scrartcl}
\pdfminorversion=5 % erlaubt das Einfügen von pdf-Dateien bis Version 1.7, ohne eine Fehlermeldung zu werfen (keine Garantie für fehlerfreies Einbetten!)

% Dokumenteninformationen ----------------------------------------------------
\newcommand{\titel}{Anforderungsanalyse}
\newcommand{\untertitel}{Studienarbeit \semester}
\newcommand{\kompletterTitel}{\titel{} \\ \untertitel}
\newcommand{\datum}{\today}

\newcommand{\vorlagenOrdner}{../../99_Vorlagen} % Falls im Unterordner ../ vorne hinzufügen

\newcommand{\betriebLogo}{\vorlagenOrdner/Bilder/logo}

% Konfiguration -------------------------------------------------------------
\newcommand{\autoren}{
    \author{
        Schmid, Mike\\
        \texttt{sgschwin@hsr.ch}
        \and
        Schlatter, Janik\\
        \texttt{jschlatt@hsr.ch}
    }
}

\newcommand{\betreuer}{
    Stettler Beat\\
    \scriptsize \texttt{\url{beat.stettler@hsr.ch}}
    \normalsize
}

\newcommand{\schmid}{
    Mike Schmid\\
    \url{mschmid@hsr.ch}
    \normalsize
}

\newcommand{\schlatter}{
    Janik Schlatter\\
    \scriptsize \url{jschlatt@hsr.ch}
    \normalsize
}

\newcommand{\autorenNamen}{
    M. Schmid, J. Schlatter
}

\newcommand{\semester}{FS-2020}
\newcommand{\betriebName}{\textsc{HSR} Hochschule für Technik Rapperswil} % Metadaten zu diesem Dokument (Autor usw.)
\input{\vorlagenOrdner/Konfiguration/Packages} % verwendete Packages
\input{\vorlagenOrdner/Konfiguration/Seitenstil_Bericht} % Definitionen zum Aussehen der Seiten
\input{\vorlagenOrdner/Konfiguration/Befehle} % eigene allgemeine Befehle, die z.B. die Arbeit mit LaTeX erleichtern

\begin{document}

% Deckblatt ------------------------------------------------------------------
\phantomsection
\thispagestyle{plain}
\pdfbookmark[1]{Deckblatt}{deckblatt}
\begin{titlepage}
    \begin{center}
        \includegraphics[scale=1.5]{\betriebLogo}\\[10ex]

        \rule{\linewidth}{0.5mm}\\[2ex]
        {\huge \bfseries  \titel }\\[2ex]
        {\LARGE \untertitel }\\[2ex]
        {\large \datum}\\
        \rule{\linewidth}{0.5mm}\\[10ex]

        \begin{minipage}[t]{0.4\textwidth}
            \begin{flushleft} 
                \large \emph{Autoren:}\\
                    \large Mike \textsc{Schmid}\\
                    \scriptsize \texttt{mike.schmid@hsr.ch}\\[1ex]
                    \large Janik \textsc{Schlatter}\\
                    \scriptsize \texttt{janik.schlatter@hsr.ch}\\[1ex]
            \end{flushleft}
            \end{minipage}
            ~
            \begin{minipage}[t]{0.4\textwidth}
            \begin{flushright} 
                \large \emph{Supervisor:} \\
                Prof. Stettler \textsc{Beat}\\
                \scriptsize \texttt{beat.stettler@hsr.ch}\\[1ex]
            \end{flushright}
        \end{minipage}\\[40ex]

        \small
        \noindent
        Dieses Werk einschließlich seiner Teile ist \textbf{urheberrechtlich geschützt}.
        Jede Verwertung außerhalb der engen Grenzen des Urheberrechtgesetzes ist ohne
        Zustimmung des Autors unzulässig und strafbar. Das gilt insbesondere für
        Vervielfältigungen, Übersetzungen, Mikroverfilmungen sowie die Einspeicherung
        und Verarbeitung in elektronischen Systemen.

    \end{center}
\end{titlepage}
\cleardoublepage

% Preface --------------------------------------------------------------------
\pagenumbering{Roman}

\section{Sinn und Zweck}
Dieses Dokument beschreibt die Personas, die Use-Cases und die nichtfunktionalen Anforderungen an die Studienarbeit Nuts 2.0
% Änderungsgeschichte
\section*{Änderungsgeschichte}
\begin{tabularx}{\textwidth}{llXl}
	\toprule
	Datum & Version & Änderung & Autor \\
	\midrule
	27.02.2020 & 1.0 & Initial Setup & Janik Schlatter \\
	\bottomrule
\end{tabularx}
\cleardoublepage

% Inhaltsverzeichnis
\phantomsection
\pdfbookmark[1]{Inhaltsverzeichnis}{inhalt}
\tableofcontents
\cleardoublepage

\pagenumbering{arabic}
% Jede Überschrift 1 auf neuer Seite
\let\stdsection\section
\renewcommand\section{\clearpage\stdsection}

% Inhalt ---------------------------------------------------------------------
\section{Use Cases}
	\subsection{Personas}
		\begin{tabularx}{\textwidth}{lXX}
			\toprule
			Person & Beschreibung & Technisches Wissen \\
			\midrule
			Net-Admin & Der Netzwerk-Administrator hat die Verantwortung über das ganze Netzwerk & Er sollte in der Lage sein, Python-Code zu interpretieren und allenfalls zu erweitern.\\
			\midrule
			Net-Engineer & Der Netzwerk-Engineer ist die Person, welche das Netwerk betreibt, er setzt Änderungen und Erweiterungen um und betreibt im Fehlerfall Troubleshooting. & Der Netzwerk-Engineer sollte fundierte Python-Kenntnisse haben und in der Lage sein, das Programm bei Bedarf zu verändern. \\
			\midrule
			Net-Techniker & Der Netzwerk-Techniker unterstützt den Netzwerk-Engineer bei der Wartung und dem Betrieb des Netzwerks. & Der Netzwerk-Techniker hat, wenn überhaupt, nur geringe Kenntnisse über Python. Er soll, ohne Code zu schreiben, dazu in der Lage sein, das Programm auszuführen. \\
			\bottomrule
		\end{tabularx}

	\subsection{Use Cases Brief}
		\subsubsection{Tests CRUD}
			Ein User kann Netzwerktests mit der definierten Sprache definieren. Er benötigt dazu Kenntnisse des Netzwerks und grundlegende Erfahrung in YAML.

		\subsubsection{Device erfassen}
			Für einen Test sind ein, oder mehrere Devices notwendig. Devices haben Eigenschaften wie Name, IP-Adresse, Device-Typ (Router, Switch,...) und Login Daten.
			Diese Devices werden in einer eigenen Sektion in der Testdefinition erfasst.

		\subsubsection{Command erfassen}
			Sobald man ein Device erfasst hat, möchte man Kommandos auf diesem ausführen.
			Dies könnten beispielsweise show-Befehle oder andere Befehle zum Senden von Daten sein. 

		\subsubsection{Ergebnis erfassen}
			Sobald Devices und Commands erfasst wurden, können nun die zu erwartete Ergebnis formuliert werden. Diese sind als Soll-Werte zu interpretieren und werden vom Programm bei der Durchführung mit den Ist-Werten verglichen.

		\subsubsection{Tests ausführen}
			Ein fertig formulierter Test kann ausgeführt werden.
			Dafür wird eine Verbindung zum Netzwerkgerät aufgebaut und das in der Testdefinition spezifizierte Kommando ausgeführt.

		\subsubsection{Logging/Reports betrachten}
			Die Auswertung der jeweiligen Durchführung der Tests wird in einem Testreport gespeichert, welchen der User jederzeit einsehen kann, um sich einen Überblich über die Historie vergangener Testdurchführungen zu verschaffen.

	\subsection{Use Case Diagramm}
		\includegraphics[scale=0.5]{\vorlagenOrdner/Bilder/UseCaseDiagram}

\section{Nichtfunktionale Anforderungen}
	In diesem Kapitel werden die nichtfunktionalen Anforderungen an das Projekt behandelt.
	Es werden Aspekte und Anforderungen aus den Bereichen Änderbarkeit, Benutzbarkeit, Effizienz, Zuverlässigkeit, Betreibbarkeit und Sicherheit betrachtet.
	Die jeweiligen Aspekte werden in ihren Unterkapiteln genauer beschrieben.
	Es werden mögliche Szenarien beschrieben, die in der Erstellung oder dem Betrieb der Software auftreten können und bei der Architektur in betracht gezogen werden.

	\subsection{Änderbarkeit}
		Aufwand, der zur Durchführung von vorgegebenen Änderungsarbeiten benötigt wird.
		Unter Änderungen gehen Korrekturen, Anpassungen oder Veränderungen der Umgebung, Anforderungen oder funktionalen Spezifikation.
		Gemäss ISO 9126 gehören zur Änderbarkeit folgende Teilmerkmale:
		
		\subsubsection{Analysierbarkeit}
			Aufwand, der benötigt wird, um das System zu verstehen, z.B. um Ursachen von Versagen oder Mängel zu diagnostizieren oder Änderungen zu planen.

		\subsubsection{Modifizierbarkeit}
			Wie leicht lässt sich das System anpassen, um Verbesserungen oder Fehlerbeseitigungen durchzuführen.

		\subsubsection{Stabilität}
			Wahrscheinlichkeit, dass mit Änderungen unerwartete Nebenwirkungen auftreten.

		\subsubsection{Testbarkeit}
			Wie gross wird der Aufwand, bei Änderungen die Software zu prüfen.

		\subsubsection{Szenario: Neue Netzwerkschnittstelle}
			Wenn zum bestehenden System eine neue Netzwerkschnittstelle definiert werden soll, so muss die dafür notwendige Software innerhalb von einer Arbeitswoche entwickelt, integriert und in Betrieg genommen werden können.

			\begin{tabularx}{\textwidth}{lX}
				\toprule
				Qualitätsziele & Flexibilität, Erweiterbarkeit, Anpassbarkeit, Austauschbarkeit  \\
				\midrule
				Geschäftsziel(e) & Software kann mit geringem Aufwand an geänderte Anforderungen angepasst werden  \\
				\midrule
				Auslöser & Es besteht eine andere Möglichkeit, auf ein Netzwerkgerät über eine Schnittstelle zuzugreifen, die nicht im System integriert und der implementierten Methode zu bevorzugen ist.  \\
				\midrule
				Reaktion & Die Software lässt sich von einem Entwickler in weniger als einer Woche um benötigte Komponenten erweitern.  \\
				\midrule
				Zielwert & 	Erweiterungen der Netzwerkschnittstellen sind innerhalb von 40 Personenstunden umsetzbar.  \\
				\bottomrule
			\end{tabularx}

		\subsubsection{Szenario: Verständlichkeit von generiertem Code}
			Generierter Code für die Netzwerktests ist leicht verständlich und manuell anpassbar.

			\begin{tabularx}{\textwidth}{lX}
				\toprule
				Qualitätsziele & Verständlichkeit, Testbarkeit, Modifizierbarkeit  \\
				\midrule
				Geschäftsziel(e) &  Tester können den generierten Code für die Testsuites und die Testfälle leicht verstehen und ihren eigenen Bedürfnissen anpassen. \\
				\midrule
				Auslöser & Ein Netzwerengineer möchte an der Software Änderungen vornehmen und dafür die Tests anpassen.  \\
				\midrule
				Reaktion & Die Testsuites und der Testcode für die Netzwerkeinstellungen werden durch einen Test-Generator in möglichst einfacher Form generiert. \\
				\midrule
				Zielwert & Tester und Entwickler können die generierten Tests in weniger als 30 Minuten verstehen und einfache Anpassungen vornehmen. \\
				\bottomrule
			\end{tabularx}

		\subsection{Scenario: Schnelle Fehlerlokalisierung}
			Die Ursache von fehlgeschlagenen Tests (Software-Unittests) lässt sich in kurzer Zeit lokalisieren.

			\begin{tabularx}{\textwidth}{lX}
				\toprule
				Qualitätsziele & Schnelle Fehlerbehebung, Änderbarkeit, Anpassbarkeit, geringes Risiko bei Erweiterungen  \\
				\midrule
				Geschäftsziel(e) & Entwickler können das Programm einfach anpassen und erkennen im Fehlerfall schnell, was nicht funktioniert hat.  \\
				\midrule
				Auslöser & Eine Änderung im Code führt zu Fehlnern in der Ausführung.  \\
				\midrule
				Reaktion & Wenn ein Fehler dazu führt, dass die Softwareausführung fehlschlägt, kann ein Entwickler aufgrund von Fehler- und/oder Log-Nachrichten die Ursache in kurzer Zeit lokalisieren.  \\
				\midrule
				Zielwert & Fehlerlokalisierung findet durchschnittlich in weniger als 10 Minuten statt.  \\
				\bottomrule
			\end{tabularx}

	\subsection{Benutzbarkeit}
		Zeitlicher Aufwand, der für die Erlernung der Benutzung des Programms benötigt wird. Die User werden hierfür in spezifische Nutzergruppen mit festgelegten Fähigkeiten unterteilt.
		
		\subsubsection{Verständlichkeit}
			Aufwand für den Nutzer, die Konzepte und Menüführung der Anwendung zu verstehen.

		\subsubsection{Erlernbarkeit}
			Aufwand für den User, sich ohne Vorwissen in das System einzuarbeiten.

		\subsubsection{Bedienbarkeit}
			Aufwand für den Benutzer, die Anwendung zu bedienen.

		\subsubsection{Szenario: Einfachheit der Testdefinitionen}
			Die Definitionen von Tests in YAML sind so aufgebaut, dass ein User in kurzer Zeit die Struktur und den Aufbau versteht und eigene Tests implementieren kann.
			
			\begin{tabularx}{\textwidth}{lX}
				\toprule
				Qualitätsziele & Produktivität, Einfachheit, Verständlichkeit \\
				\midrule
				Geschäftsziel(e) & Einarbeitung in die Testdefinition erfolg möglichst einfach und benötigt nur geringes Vorwissen.  \\
				\midrule
				Auslöser & Ein Nutzer, welcher keine Erfahrung im Umgang mit der Software hat, möchte eigene Tests definieren.  \\
				\midrule
				Reaktion & Benutzer können sich schnell in die Testdefinitionen einlesen und rasch eigene Tests definieren, vorausgesetzt, sie haben Kenntnisse des Netzwerkes.  \\
				\midrule
				Zielwert & Ungeschulte Nutzer verstehen innerhalb von durchschnittlich 30 Minuten die Struktur und den Aufbau der Testdefinitionen und sind in der Lage, eigene Tests zu erstellen.  \\
				\bottomrule
			\end{tabularx}
			
		\subsubsection{Szenario: Hinweis auf Fehleingaben}
		Fehlerhafte Eingaben werden vom System ignoriert und der Benutzer wird auf die falsche Eingabe hingewiesen. Das Programm führt fehlerfreie Programmteile unabhängig von den Fehlern durch.
			
		\begin{tabularx}{\textwidth}{lX}
			\toprule
			Qualitätsziele & Robustheit, Verständlichkeit, Fehlertoleranz.  \\
			\midrule
			Geschäftsziel(e) & Fehleingaben führen nicht dazu, dass die Tests nicht mehr durchgeführt werden können.  \\
			\midrule
			Auslöser & Ein Benutzer macht einen Fehler bei der Testdefinition und startet das Programm.  \\
			\midrule
			Reaktion & Das Programm führt alle korrekten Tests durch und informiert den Benutzer, dass es fehlerhafte Tests gibt, die nicht durchgeführt werden können. Die Hinweise werden im Report und auf der Konsolenausgabe geschrieben.   \\
			\midrule
			Zielwert & Tests sind einzeln gekapselt und werden unabhängig voneinander durchgeführt. Falscheingaben werden vom Programm detektiert und im Testreport sowie auf der Konsolenausgabe erwähnt.  \\
			\bottomrule
		\end{tabularx}
		
	\subsection{Effizienz}
	Mit Effizienz ist die 'performance efficiency' gemeint, d.h. das Verhältnis zwischen dem Leistungsniveau der Software und den eingesetzten Hardwarekomponenten. 
	Andere Beschreibungen umfassen: Skalierbarkeit, Speicherbedarf, Verarbeitungsgeschwindigkeit, Antwortzeit etc.
	Teilmerkmale nach ISO 9126:

		\subsubsection{Zeitverhalten}
		Dauer für Verarbeitung und Antwortzeit sowie Durchsatz bei der Ausführung des Programms

		\subsubsection{Verbrauchsverhalten}
		Wie viel Speicherbedarf hat das Programm, wie lange werden Betriebsmittel in Anspruch genommen und welche Hardwarekomponenten werden benötigt.

		\subsubsection{Szenario: Schnelle Erzeugung der Netztestdaten}
		Nach der Erstellung von Testdefinitionen können die Netztests ohne lange Wartezeiten erstellt werden um eine rasche Arbeitsweise zu garantieren.
			
		\begin{tabularx}{\textwidth}{lX}
				\toprule
				Qualitätsziele & Performanz, Laufzeitverhalten, Flexibilität, Geschwindigkeit  \\
				\midrule
				Geschäftsziel(e) & Die Zeit zwischen der Erstellung und der Durchführung von Tests wird gering gehalten.  \\
				\midrule
				Auslöser & Ein Benutzer erstellt Netztests und möchte diese rasch in das System einspeisen.  \\
				\midrule
				Reaktion & Tests werdem vom Testgenerator erstellt, so dass die Wartezeiten für Benutzer möglichst gering sind.  \\
				\midrule
				Zielwert & Netztests werden in weniger als 10 Sekunden generiert.  \\
				\bottomrule
			\end{tabularx}

		\subsubsection{Szenario: Optimierte Durchführung von Tests}
		Die Tests werden möglichst parallel abgearbeitet und nur dann seriell durchgeführt, wenn eine asynchrone Ausführung zu Störungen im Netzwerk führen würden. 
		
		\begin{tabularx}{\textwidth}{lX}
			\toprule
			Qualitätsziele & Effizienz, geringe Störung im zu testenden Netzwerk, Robustheit  \\
			\midrule
			Geschäftsziel(e) & Die Durchführung von Netzwerktests führt nicht zu Performanzeinbussen im Netzwerk.  \\
			\midrule
			Auslöser & Es werden mehrere parallelisierbare (z.B. Ping) und mehrere performanzstörende (z.B. Traffic-Test) Tests definiert.  \\
			\midrule
			Reaktion & Parallelisierbare Tests werden asynchron durchgeführt. Alle Tests, die das Netzwerk stören können werden nacheinander abgearbeitet. \\
			\midrule
			Zielwert & Es entstehen maximal 30\% Performanzeinbussen im Netzwerk während die Tests durchgeführt werden.  \\
			\bottomrule
		\end{tabularx}

	\subsection{Zuverlässigkeit}
	Unter Zuverlässigkeit versteht man die Fähigkeit der Software, unter festgelegten Bedingungen die Funktionalität über einen definierten Zeitraum zu gewährleisten
	
		\subsubsection{Reife}
		Geringe Ausfallhäufigkeit durch Fehlzustände.

		\subsubsection{Fehlertoleranz}
		Die Software ist in der Lage, trotz Fehlern ihr spezifiziertes Leistungsniveau beizubehalten.

		\subsubsection{Wiederherstellbarkeit}
		Im Fehlerfall können betroffene Daten wiederhergestellt und die Funktionalität wieder aufgenommen werden.

		\subsubsection{Szenario: Tests lassen sich auf der Netzwerkseite nicht ausführen}
		Falls ein Test auf dem jeweiligen Netzwerkgerät nicht erfolgreich durchgeführt werden kann, läuft das Programm weiter und definiert den dazugehörigen Netzwerktest als nicht bestanden.
		
		\begin{tabularx}{\textwidth}{lX}
			\toprule
			Qualitätsziele & Robustheit, Behandlung Infrastrukturbedingter Fehler.  \\
			\midrule
			Geschäftsziel(e) & Das System führt alle Tests unabhängig voneinander durch. Wenn ein Test zu einem Fehler führt, weil z.B. ein falsches Netzwerkgerät angegeben wurde, wird dieser Test unabhängig von allen anderen Tests fehlschlagen.  \\
			\midrule
			Auslöser & Test lässt sich auf spezifizierter Infrastruktur nicht ausführen.  \\
			\midrule
			Reaktion & Test schlägt fehl und mögliche Ursachen werden im Report und in der Konsole angezeigt. Alle anderen Tests laufen durch.  \\
			\midrule
			Zielwert & Das Fehlschlagen eines Tests fürht nicht zum Programmabbruch.  \\
			\bottomrule
		\end{tabularx}

	\subsection{Betreibbarkeit}
	Die Betriebbarkeit wird in der ISO 9126 nicht definiert. Die ISO spezifiziert aber mehrere Teilmerkmale, die unter dem Begriff Betreibbarkeit zusammengefasst werden können:

		\subsubsection{Analysierbarkeit}
		Aufwand, der benötigt wird, um den Code zu analysieren, um im falle eines Versagens dessen Ursachen zu diagnostizieren oder um Änderungen zu planen und durchzuführen.

		\subsubsection{Installierbarkeit}
		Aufwand, das Programm auf einem frisch aufgesetzten Gerät laufen zu lassen.

		\subsubsection{Übertragbarkeit}
		Kann die Software von einer Umgebung auf eine andere übertragen werden. 
		Als Umgebung zählen Hardwarekomponenten, Softwarekomponenten, Organisatorische Umgebungen oder Betriebssysteme. 

		\subsubsection{Austauschbarkeit}
		Aufwand und Möglichkeit, die Software anstelle einer anderen in deren spezifizierten Umgebung laufen zu lassen.

		\subsubsection{Koexistenz}
		Fähigkeit der Software, neben anderen Programmen mit ähnlichen oder übereinstimmenden Funktionen zu arbeiten.

		\subsubsection{Szenario: Einfache Installation auf einem neuen Gerät}
		Das Programm lässt sich auf einem neuen Gerät ohne grossen Mehraufwand installieren, ohne dass die Funktionalität des Geräts beeinflusst wird.

		\begin{tabularx}{\textwidth}{lX}
			\toprule
			Qualitätsziele & Einfachheit, Portierbarkeit, Benutzbarkeit  \\
			\midrule
			Geschäftsziel(e) & Die Installation der Software ist so einfach, dass sie innert kurzer Zeit und/oder automatisiert durchgeführt werden kann.  \\
			\midrule
			Auslöser & Die Testsoftware soll auf einem frisch aufgesetzten Gerät installiert werden.  \\
			\midrule
			Reaktion & Installationszeiten sind gering, benötigen wenige bis keine weiteren Softwarekomponenten oder lässt sich mit einigen Kommandozeilenbefehlen automatisch installieren.  \\
			\midrule
			Zielwert & Die Software wird mit einer Installationsanleitung ausgeliefert, die einfach und verständlich die Inbetriebnahme des Programms erklärt. Abhängigkeiten zu anderen Softwarekomponenten werden bewusst gering gehalten um eine einfache Installation mit weniger als 30 Minuten Zeitaufwand zu gewährleisten.  \\
			\bottomrule
		\end{tabularx}


	\subsection{Sicherheit}
	In dieser Sektion werden Sicherheitsanforderungen beschrieben. 
	Verschlüsselung, Privacy und der Umgang mit Passwörtern.

		\subsubsection{Verschlüsselung von Datenübertragungen}
		Die Netzwerktest werden über eine Verschlüsselte Verbindung durchgeführt, die dem aktuellen Stand der Technik entspricht.

		\subsubsection{Umgang mit Passwörtern}
		\textcolor{red}{Zum Zeitpunkt der Erstellung dieses Dokuments ist noch nicht klar, wie mit Passwörtern sicher umgegangen werden soll. Die Usability setzt voraus, dass der Testdurchführer nicht für jeden Test einzeln ein Passwort eingeben muss, Security-verhaltensweisen verbieten aber das Abspeichern von Passwörtern als lesbaren Text.}

\end{document}