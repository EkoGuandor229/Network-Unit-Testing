\documentclass[
	ngerman,
	toc=listof, % Abbildungsverzeichnis sowie Tabellenverzeichnis in das Inhaltsverzeichnis aufnehmen
	toc=bibliography, % Literaturverzeichnis in das Inhaltsverzeichnis aufnehmen
	footnotes=multiple, % Trennen von direkt aufeinander folgenden Fußnoten
	parskip=half, % vertikalen Abstand zwischen Absätzen verwenden anstatt horizontale Einrückung von Folgeabsätzen
	numbers=noendperiod % Den letzten Punkt nach einer Nummerierung entfernen (nach DIN 5008)
]{scrartcl}
\pdfminorversion=5 % erlaubt das Einfügen von pdf-Dateien bis Version 1.7, ohne eine Fehlermeldung zu werfen (keine Garantie für fehlerfreies Einbetten!)

% Dokumenteninformationen ----------------------------------------------------
\newcommand{\titel}{Anforderungsanalyse}
\newcommand{\untertitel}{Studienarbeit \semester}
\newcommand{\kompletterTitel}{\titel{} \\ \untertitel}
\newcommand{\datum}{\today}

\newcommand{\vorlagenOrdner}{../../99_Vorlagen} % Falls im Unterordner ../ vorne hinzufügen

\newcommand{\betriebLogo}{\vorlagenOrdner/Bilder/logo}

% Konfiguration -------------------------------------------------------------
\newcommand{\autoren}{
    \author{
        Schmid, Mike\\
        \texttt{sgschwin@hsr.ch}
        \and
        Schlatter, Janik\\
        \texttt{jschlatt@hsr.ch}
    }
}

\newcommand{\betreuer}{
    Stettler Beat\\
    \scriptsize \texttt{\url{beat.stettler@hsr.ch}}
    \normalsize
}

\newcommand{\schmid}{
    Mike Schmid\\
    \url{mschmid@hsr.ch}
    \normalsize
}

\newcommand{\schlatter}{
    Janik Schlatter\\
    \scriptsize \url{jschlatt@hsr.ch}
    \normalsize
}

\newcommand{\autorenNamen}{
    M. Schmid, J. Schlatter
}

\newcommand{\semester}{FS-2020}
\newcommand{\betriebName}{\textsc{HSR} Hochschule für Technik Rapperswil} % Metadaten zu diesem Dokument (Autor usw.)
\input{\vorlagenOrdner/Konfiguration/Packages} % verwendete Packages
\input{\vorlagenOrdner/Konfiguration/Seitenstil_Bericht} % Definitionen zum Aussehen der Seiten
\input{\vorlagenOrdner/Konfiguration/Befehle} % eigene allgemeine Befehle, die z.B. die Arbeit mit LaTeX erleichtern

\begin{document}

% Deckblatt ------------------------------------------------------------------
\phantomsection
\thispagestyle{plain}
\pdfbookmark[1]{Deckblatt}{deckblatt}
\begin{titlepage}
    \begin{center}
        \includegraphics[scale=1.5]{\betriebLogo}\\[10ex]

        \rule{\linewidth}{0.5mm}\\[2ex]
        {\huge \bfseries  \titel }\\[2ex]
        {\LARGE \untertitel }\\[2ex]
        {\large \datum}\\
        \rule{\linewidth}{0.5mm}\\[10ex]

        \begin{minipage}[t]{0.4\textwidth}
            \begin{flushleft} 
                \large \emph{Autoren:}\\
                    \large Mike \textsc{Schmid}\\
                    \scriptsize \texttt{mike.schmid@hsr.ch}\\[1ex]
                    \large Janik \textsc{Schlatter}\\
                    \scriptsize \texttt{janik.schlatter@hsr.ch}\\[1ex]
            \end{flushleft}
            \end{minipage}
            ~
            \begin{minipage}[t]{0.4\textwidth}
            \begin{flushright} 
                \large \emph{Supervisor:} \\
                Prof. Stettler \textsc{Beat}\\
                \scriptsize \texttt{beat.stettler@hsr.ch}\\[1ex]
            \end{flushright}
        \end{minipage}\\[40ex]

        \small
        \noindent
        Dieses Werk einschließlich seiner Teile ist \textbf{urheberrechtlich geschützt}.
        Jede Verwertung außerhalb der engen Grenzen des Urheberrechtgesetzes ist ohne
        Zustimmung des Autors unzulässig und strafbar. Das gilt insbesondere für
        Vervielfältigungen, Übersetzungen, Mikroverfilmungen sowie die Einspeicherung
        und Verarbeitung in elektronischen Systemen.

    \end{center}
\end{titlepage}
\cleardoublepage

% Preface --------------------------------------------------------------------
\pagenumbering{Roman}

% Änderungsgeschichte
\section*{Änderungsgeschichte}
\begin{tabularx}{\textwidth}{llXl}
	\toprule
	Datum & Version & Änderung & Autor \\
	\midrule
	27.02.2020 & 1.0 & Initial Setup & Janik Schlatter \\
	\bottomrule
\end{tabularx}
\cleardoublepage

% Inhaltsverzeichnis
\phantomsection
\pdfbookmark[1]{Inhaltsverzeichnis}{inhalt}
\tableofcontents
\cleardoublepage

\pagenumbering{arabic}
% Jede Überschrift 1 auf neuer Seite
\let\stdsection\section
\renewcommand\section{\clearpage\stdsection}

% Inhalt ---------------------------------------------------------------------
\section{Allgemeine Beschreibung}

\section{Use Cases}
	\subsection{Personas}
		\begin{tabularx}{\textwidth}{lXX}
			\toprule
			Person & Beschreibung & Technisches Wissen \\
			\midrule
			Net-Admin & Der Netzwerk-Administrator ist der Chef über das ganze Netzwerk und trägt die ganze Verantwortung darüber. & Der Netzwerk-Administrator hat grundlegendes Wissen über Python.\\
			\midrule
			Net-Engineer & Der Netzwerk-Engineer ist die Person, welche das Netwerk aufgesetzt hat und es am laufen hält. & Der Netzwerk-Engineer hat wissen über Python und sollte im Stande sein, das Programm in Betrieb zu nehmen. \\
			\midrule
			Net-Techniker & Der Netzwerk-Techniker unterstützt den Netzwerk-Engineer bei der Wartung und erledigt den Support. & Der Netzwerk-Techniker hat kein wissen über Python. Er ist nur im Stande das Programm zu benutzen, aber nicht aufzusetzen. \\
			\bottomrule
		\end{tabularx}

	\subsection{Use Cases Brief}
		\subsubsection{Tests CRUD}
			Der User kann Tests mit der definierten Sprache erstellen.

		\subsubsection{Device erfassen}
			Für einen Test sind ein- oder mehrere Devices notwendig. Devices haben Eigenschaften
			wie Beispielsweise Name, IP-Adresse, Device-Typ (Router, Switch,...) und Login Daten.
			Diese Devices müssen mittels einem Setup definiert werden.

		\subsubsection{Command erfassen}
			Sobald man ein Device erfasst hat, möchte man Kommandos auf diesem ausführen.
			Dies könnten beispielsweise show-Befehle oder andere Programme zum Senden von
			Daten sein. Diese Kommandos sind als Queries auf die Devices zu verstehen.

		\subsubsection{Ergebnis erfassen}
			Sobald Devices und Commands erfasst wurden, kann nun das 
			erwartete Ergebnis formuliert werden.

		\subsubsection{Tests ausführen}
			Ein fertig formulierter Test kann ausgeführt werden. Mit diesem Vorgang wird die
			Verbindung zum Device aufgebaut und das im Test definierte Kommando ausgeführt.

		\subsubsection{Logs}
			Die Auswertung der jeweiligen Durchführung der Tests wird in einem Logging gespeichert, welches der User jederzeit ansehen kann.

	\subsection{Use Case Diagramm}
		\includegraphics[scale=0.5]{\vorlagenOrdner/Bilder/UseCaseDiagram}

\section{Nichtfunktionale Anforderungen}
	In diesem Kapitel behandeln wir die nichtfunktionalen Anforderungen an das Projekt.
	Es werden Aspekte und Anforderungen aus den Bereichen Änderbarkeit, Benutzbarkeit, Effizienz, Zuverlässigkeit, Betriebbarkeit und Sonstige Qualitätsanforderungen.
	Die jeweiligen Aspekte werden in ihren Unterkapiteln genauer beschrieben.
	Es werden mögliche Szenarien beschrieben, die in der Erstellung oder dem Betrieb der Software auftreten können.

	\subsection{Änderbarkeit}
		Aufwand, der zur Durchführung von vorgegebenen Änderungsarbeiten benötigt wird.
		Unter Änderungen gehen Korrekturen, Anpassungen oder Veränderungen der Umgebung, Anforderungen oder funktionalen Spezifikation.
		Gem. ISO 9126 gehören zur Änderbarkeit folgende Teilmerkmale:
		\subsubsection{Analysierbarkeit}
			Aufwand, der benötigt wird, um das System zu verstehen, z.B. um Ursachen von Versagen oder Mängel zu diagnostizieren oder Änderungen zu spezifizieren.

		\subsubsection{Modifizierbarkeit}
			Wie leicht lässt sich das System anpassen, um Verbesserungen oder Fehlerbeseitigungen durchzuführen.

		\subsubsection{Stabilität}
			Wahrscheinlichkeit, dass mit Änderungen unerwartete Nebenwirkungen auftreten.

		\subsubsection{Testbarkeit}
			Wie gross wird der Aufwand, bei Änderungen die Software zu prüfen.

		\subsubsection{Szenario: Neue Netzwerkschnittstelle}
			Wenn zum bestehenden System eine neue Netzwerkschnittstelle definiert werden soll, so muss die dafür notwendige Software innerhalb von einer Arbeitswoche entwickelt, integriert und in Betrieg genommen werden können.

			\minisec{Geschäftsziele}
				Flexibilität und Reaktionsfähigkeit bei Änderungswünschen

			\minisec{Auslöser}
				Entwickler möchte für eine nicht vorhandene Netzwerkschnittstelle eine Erweiterung erstellen, z.B. Einbindung von OpenConfig.

			\minisec{Reaktion}
				Betreiber der Software können gem. ihren eigenen Anforderungen an die Netzwerkschnittstellen auf Basis vordefinierter Erweiterungspunkte das System erweitern.

			\minisec{Zielwert}
				Erweiterungen der Netzwerkschnittstellen sind innerhalb von 40 Personenstunden umsetzbar

		\subsubsection{Szenario: Verständlichkeit von generiertem Code}
			Generierter Code für die Testsuites und Testfälle ist leicht verständlich und manuell anpassbar.

			\minisec{Qualitätsziele}
				Verständlichkeit, Testbarkeit, Modifizierbarkeit

			\minisec{Geschäftsziel(e)}
				Tester können den generierten Code für die Testsuites und die Testfälle leicht verstehen und ihren eigenen Bedürfnissen anpassen.

			\minisec{Reaktion}
				Testcode (Python pytest-Code) für bestimmte Teile des Systems wird durch den Test-Generator in möglichst einfacher Form erzeugt.

			\minisec{Zielwert}
				Ein Tester kann den generieten Quellcode fü einen Testfall in weniger als 30 Minuten verstehen und einfache Änderungen daran vornehmen.

		\subsection{Scenario: Schnelle Fehlerlokalisierung}
			Die Ursache von fehlgeschlagenen Tests (Software-Unittests) lässt sich in kurzer Zeit lokalisieren.

			\minisec{Geschäftsziele}
				Kurze Fehlerbehebungszeiten, Änderbarkeit, geringes Risiko bei Änderungen

			\minisec{Reaktion}
				Wenn ein Test fehlschlägt, kann ein Entwickler aufgrund von Fehler- und/oder Log-Nachrichten die Ursache in kurzer Zeit lokalisieren.

			\minisec{Zielwert}
				Fehlerlokalisierung findet durchschnittlich in unter 10 Minuten statt.
	
	\subsection{Benutzbarkeit}
		Zeitlicher Aufwand, der für die Erlernung der Benutzung des Programms benötigt wird. Die User werden hierfür in spezifische Nutzergruppen mit festgelegten Fähigkeiten unterteilt.
		Teilmerkmale:
		
		\subsubsection{Verständlichkeit}
			Aufwand für den Nutzer, die Konzepte und Menüführung der Anwendung zu verstehen.

		\subsubsection{Erlernbarkeit}
			Aufwand für den User, sich ohne Vorwissen in das System einzuarbeiten.

		\subsubsection{Bedienbarkeit}
			Aufwand für den Benutzer, die Anwendung zu bedienen.

		\subsubsection{Szenario: Einfachheit der Testdefinitionen}
			Die Definitionen von Tests in YAML sind so aufgebaut, dass ein User in kurzer Zeit die Struktur und den Aufbau versteht und eigene Tests implementieren kann.
			\minisec{Geschäftsziele}
			\textcolor{red}{TODO FINISH ZHIS}


	\subsection{Sicherheit}
	Um auf Devices verbinden zu können, muss man sich auf diesen authentifizieren. Administrator Zugänge müssen deshalb best möglichst geschützt sein. Die Übertragung
	muss verschlüsselt sein (SSH) und die Passwörter dürfen, wenn überhaupt, nur mittels
	sicherem Hashverfahren abgelegt werden. \textcolor{red}{TODO Update}

\section{Weitere Anforderungen}
	\subsection{Schnittstellen}
		NUTS hat verschiedene interne Schnittstellen, welche hier aufgezeigt werden: \\
		\begin{tabularx}{\textwidth}{lX}
			\toprule
			Schnittstelle & Beschriebung \\
			\midrule
			Benutzerschnittstelle & \textcolor{red}{TODO} \\
			Netzwerkschnittstelle & \textcolor{red}{TODO} \\
			\bottomrule
		\end{tabularx}

	\subsection {Randbedingungen}

\end{document}