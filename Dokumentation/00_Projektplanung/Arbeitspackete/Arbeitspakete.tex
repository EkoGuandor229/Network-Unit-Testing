\documentclass[
	ngerman,
	toc=listof, % Abbildungsverzeichnis sowie Tabellenverzeichnis in das Inhaltsverzeichnis aufnehmen
	toc=bibliography, % Literaturverzeichnis in das Inhaltsverzeichnis aufnehmen
	footnotes=multiple, % Trennen von direkt aufeinander folgenden Fußnoten
	parskip=half, % vertikalen Abstand zwischen Absätzen verwenden anstatt horizontale Einrückung von Folgeabsätzen
	numbers=noendperiod % Den letzten Punkt nach einer Nummerierung entfernen (nach DIN 5008)
]{scrartcl}
\pdfminorversion=5 % erlaubt das Einfügen von pdf-Dateien bis Version 1.7, ohne eine Fehlermeldung zu werfen (keine Garantie für fehlerfreies Einbetten!)

% Dokumenteninformationen ----------------------------------------------------
\newcommand{\titel}{Arbeitspakete}
\newcommand{\untertitel}{Studienarbeit \semester}
\newcommand{\kompletterTitel}{\titel{} \\ \untertitel}
\newcommand{\datum}{\today}

\newcommand{\vorlagenOrdner}{../../99_Vorlagen} % Falls im Unterordner ../ vorne hinzufügen

\newcommand{\betriebLogo}{\vorlagenOrdner/Bilder/logo}

% Konfiguration -------------------------------------------------------------
\newcommand{\autoren}{
    \author{
        Schmid, Mike\\
        \texttt{sgschwin@hsr.ch}
        \and
        Schlatter, Janik\\
        \texttt{jschlatt@hsr.ch}
    }
}

\newcommand{\betreuer}{
    Stettler Beat\\
    \scriptsize \texttt{\url{beat.stettler@hsr.ch}}
    \normalsize
}

\newcommand{\schmid}{
    Mike Schmid\\
    \url{mschmid@hsr.ch}
    \normalsize
}

\newcommand{\schlatter}{
    Janik Schlatter\\
    \scriptsize \url{jschlatt@hsr.ch}
    \normalsize
}

\newcommand{\autorenNamen}{
    M. Schmid, J. Schlatter
}

\newcommand{\semester}{FS-2020}
\newcommand{\betriebName}{\textsc{HSR} Hochschule für Technik Rapperswil} % Metadaten zu diesem Dokument (Autor usw.)
\input{\vorlagenOrdner/Konfiguration/Packages} % verwendete Packages
\input{\vorlagenOrdner/Konfiguration/Seitenstil_Bericht} % Definitionen zum Aussehen der Seiten
\input{\vorlagenOrdner/Konfiguration/Befehle} % eigene allgemeine Befehle, die z.B. die Arbeit mit LaTeX erleichtern

\begin{document}

% Deckblatt ------------------------------------------------------------------
\phantomsection
\thispagestyle{plain}
\pdfbookmark[1]{Deckblatt}{deckblatt}
\begin{titlepage}
    \begin{center}
        \includegraphics[scale=1.5]{\betriebLogo}\\[10ex]

        \rule{\linewidth}{0.5mm}\\[2ex]
        {\huge \bfseries  \titel }\\[2ex]
        {\LARGE \untertitel }\\[2ex]
        {\large \datum}\\
        \rule{\linewidth}{0.5mm}\\[10ex]

        \begin{minipage}[t]{0.4\textwidth}
            \begin{flushleft} 
                \large \emph{Autoren:}\\
                    \large Mike \textsc{Schmid}\\
                    \scriptsize \texttt{mike.schmid@hsr.ch}\\[1ex]
                    \large Janik \textsc{Schlatter}\\
                    \scriptsize \texttt{janik.schlatter@hsr.ch}\\[1ex]
            \end{flushleft}
            \end{minipage}
            ~
            \begin{minipage}[t]{0.4\textwidth}
            \begin{flushright} 
                \large \emph{Supervisor:} \\
                Prof. Stettler \textsc{Beat}\\
                \scriptsize \texttt{beat.stettler@hsr.ch}\\[1ex]
            \end{flushright}
        \end{minipage}\\[40ex]

        \small
        \noindent
        Dieses Werk einschließlich seiner Teile ist \textbf{urheberrechtlich geschützt}.
        Jede Verwertung außerhalb der engen Grenzen des Urheberrechtgesetzes ist ohne
        Zustimmung des Autors unzulässig und strafbar. Das gilt insbesondere für
        Vervielfältigungen, Übersetzungen, Mikroverfilmungen sowie die Einspeicherung
        und Verarbeitung in elektronischen Systemen.

    \end{center}
\end{titlepage}
\cleardoublepage

% Preface --------------------------------------------------------------------
\pagenumbering{Roman}

% Zweck
\section*{Zweck}
Dieses Dokument beschreibt die Arbeitspakete und deren Aufwandschätzung und Priorisierung.

% Änderungsgeschichte
\section*{Änderungsgeschichte}
\begin{tabularx}{\textwidth}{llXl}
	\toprule
	Datum & Version & Änderung & Autor \\
	\midrule
	31.03.2018 & 1.0 & Initial Setup & Janik Schlatter \\
	31.03.2020 & 1.0 & Erstellen Arbeitspakete & Mike Schmid \\
	\bottomrule
\end{tabularx}
\cleardoublepage

% Inhaltsverzeichnis
\phantomsection
\pdfbookmark[1]{Inhaltsverzeichnis}{inhalt}
\tableofcontents
\cleardoublepage

\pagenumbering{arabic}
% Jede Überschrift 1 auf neuer Seite
\let\stdsection\section
\renewcommand\section{\clearpage\stdsection}

% Inhalt ---------------------------------------------------------------------
\section{Arbeitspakete}
Die Arbeitspakete sind in fünf Kategorien eingeteilt, um eine Übersicht zu bieten. 
\begin{itemize}
	\item Ausführung der Tests - Alles was mit der Ausführung und Evaluation der Netzwerktests zu tun hat.
	\item Erstellen der Tests - Alle Arbeitspakete, die sich mit der Erstellung der Testsuite auseinandersetzen.
	\item Hilfsklassen - Filehandler, Logging, Integration Tests und Kommandozeileninteraktion
	\item Tests - Integration einzelner Tests in das System
	\item Connections - Integration weiterer Verbindungsschnittstellen in das System
\end{itemize}

	\subsection{Kategorie Ausführung der Tests}
	
	
	\subsubsection{AP: Implementation Testrunner}
	\begin{tabularx}{\textwidth}{lX}
		\toprule
		Beschreibung & Die Testrunner-Klasse führt die ihm vom Controller übergebenen Tests gegen das Netzwerk aus und gibt ein Resultat in einem beliebigen Format zurück.\\
		\midrule
		Akzeptanzkriterien & Testrunner ist in der Lage, Tests gegen ein Netzwerk auszuführen und reagiert auf Fehler, indem fehlgeschlagene Tests als solche annotiert werden.\\
		 & Es existieren Unittests für die Testrunner-Klasse und bekannte Bugs sind als weitere Arbeitspakete erfasst. \\
		\midrule
		Aufwandschätzung & 6 Stunden\\
		\midrule
		Priorität & Mittel \\
		\bottomrule
	\end{tabularx}

	\subsubsection{AP: Implementation Evaluator}
	\begin{tabularx}{\textwidth}{lX}
		\toprule
		Beschreibung & Der Evaluator vergleicht die Resultate der Netzwerktests mit den aus den Testdefinitionen beschriebenen Soll-Werten.\\
		\midrule
		Akzeptanzkriterien & Der Evaluator ist in der Lage, die normalisierten Testresultate mit den Testexpectations zu vergleichen und ein verständliches Evaluationsresultat zu erzeugen.\\
		 & Unittests sind erstellt und bekannte Bugs sind dokumentiert. \\
		\midrule
		Aufwandschätzung & 10 Stunden\\
		\midrule
		Priorität & Mittel \\
		\bottomrule
	\end{tabularx}
	\newpage

	\subsubsection{AP:Implementation Reporter}
	\begin{tabularx}{\textwidth}{lX}
		\toprule
		Beschreibung & Der Reporter nimmt die Gesamtergebnisse entgegen und gibt sie auf der Konsole in einem einfach verständlichen Format aus. \\
		& Zudem schreibt er die Ergebnisse in ein Report-File, welches die Testresultate und allfällige Fehlermeldungen beinhaltet. \\
		\midrule
		Akzeptanzkriterien & Die Reporter-Klasse ist in der Lage, detaillierte Berichte zu den Ergebnissen zu verfassen und abzuspeichern.\\
		 & Unittests sind erstellt und laufen alle durch. Bekannte Bugs sind im Issue-Tracker erfasst.\\
		\midrule
		Aufwandschätzung & 10 Stunden\\
		\midrule
		Priorität & Niedrig \\
		\bottomrule
	\end{tabularx}

	\subsubsection{AP: Implementation Testresultat-Mapper}
	\begin{tabularx}{\textwidth}{lX}
		\toprule
		Beschreibung & Der Testresultatmapper normalisiert die Rückgabewerte der einzelnen Netzwerktests in ein einheitliches Format, so dass der Evaluator damit arbeiten kann.\\
		\midrule
		Akzeptanzkriterien & Die Normalform der Resultate ist festgelegt und implementiert. Unittests beenden mit Status "Grün" und bekannte Bugs sind erfasst.\\
		\midrule
		Aufwandschätzung & 12 Stunden\\
		\midrule
		Priorität & Mittel \\
		\bottomrule
	\end{tabularx}

	\subsubsection{AP: Implementation Testcontroller}
	\begin{tabularx}{\textwidth}{lX}
		\toprule
		Beschreibung & Der Testcontroller ist der Einstiegspunkt in das Programm. Er instanziert die benötigten Klassen und steuert den Programmfluss.\\
		\midrule
		Akzeptanzkriterien & Testcontroller ist implementiert und das Programm lässt sich erfolgreich ausführen.\\
		 & Bekannte Bugs sind erfasst und Unittests sind erstellt und bestanden.\\
		\midrule
		Aufwandschätzung & 4 Stunden \\
		\midrule
		Priorität & Niedrig \\
		\bottomrule
	\end{tabularx}
	\newpage

	\subsection{Kategorie Erstellen der Tests}

	\subsubsection{AP: Implementation Testbuilder}
	\begin{tabularx}{\textwidth}{lX}
		\toprule
		Beschreibung & Der Testbuilder übernimmt die Tests aus der Testdefinition und erstellt ein Testbundle mit den angegebenen Parameter. \\
		\midrule
		Akzeptanzkriterien & Der Testbuilder ist implementiert und erstellt ausführbare Tests.\\
		 & Unittests sind implementiert und bekannte Bugs sind dokumentiert. \\
		\midrule
		Aufwandschätzung & 6 Stunden\\
		\midrule
		Priorität & Mittel\\
		\bottomrule
	\end{tabularx}

	\subsubsection{AP: Auswahl der Testreihenfolge}
	\begin{tabularx}{\textwidth}{lX}
		\toprule
		Beschreibung & Erweiterung des Testbuilder um die Möglichkeit, die Reihenfolge der Tests zu bestimmen.\\
		\midrule
		Akzeptanzkriterien & Testdurchführungsreihenfolge lässt sich ändern, ohne dass die Unittests dabei durchfallen. \\
		\midrule
		Aufwandschätzung & 6 Stunden\\
		\midrule
		Priorität & Niedrig\\
		\bottomrule
	\end{tabularx}

	\subsubsection{AP: Inventar Management Klassen}
	\begin{tabularx}{\textwidth}{lX}
		\toprule
		Beschreibung & Zu diesem Arbeitspacket gehören die Klassen Inventar, Device und DeviceConnection. Diese Klassen beschäftigen sich damit, die Informationen über die Physischen Geräte in einem Netzwerk festzuhalten.\\
		\midrule
		Akzeptanzkriterien & Klassen sind implementiert und die Zusammenarbeit funktioniert. Es lassen sich Tests mit den Gerätedaten vom TestBuilder erstellen und durchführen.\\
		\midrule
		Aufwandschätzung & 10 Stunden\\
		\midrule
		Priorität & Hoch \\
		\bottomrule
	\end{tabularx}
	\newpage

	\subsubsection{AP:Testdefinition Klassen}
	\begin{tabularx}{\textwidth}{lX}
		\toprule
		Beschreibung & Die Testdefinitionen beschreiben die auszuführenden Tests gegen das Netzwerk. Der Loader instanziert die Definitionen der einzelnen Tests und gibt diese als Collection an den Testbuilder weiter, der daraus konkrete Tests instanziert.\\
		\midrule
		Akzeptanzkriterien & Testdefinitionen lassen sich erstellen und ausführen.\\
		 & Unittests sind implementiert und laufen erfolgreich durch. Allfällige Bugs sind im Issue-Tracking dokumentiert.\\
		\midrule
		Aufwandschätzung & 8 Stunden\\
		\midrule
		Priorität & Hoch\\
		\bottomrule
	\end{tabularx}

	\subsubsection{AP: Connection-Strategy-Factory}
	\begin{tabularx}{\textwidth}{lX}
		\toprule
		Beschreibung & Dieser Teil der Software kümmert sich um die Auswahl und Instanzierung der Kommunikationsschnittstelle.\\
		\midrule
		Akzeptanzkriterien & Alle Klassen sind gemäss der Architektur implementieren und es lässt sich mindestens eine Schnittstelle auswählen und instanzieren.\\
		 & Tests sind erfolgreich und allfällige Bugs sind dokumentiert.\\
		\midrule
		Aufwandschätzung & 20 Stunden\\
		\midrule
		Priorität & Hoch\\
		\bottomrule
	\end{tabularx}

	\subsubsection{AP: Testerstellung-Strategy-Factory}
	\begin{tabularx}{\textwidth}{lX}
		\toprule
		Beschreibung & Dieser Teil des Programms entscheidet, welche konkreten Testklassen verwendet werden und instanziert diese.\\
		\midrule
		Akzeptanzkriterien & Alle Klassen sind gemäss der Architektur implementiert und es gibt mindestens eine konkrete Testklasse die sich ausführen lässt. \\
		 & Es existieren Unittests, welche erfolgreich durchlaufen und bekannte Bugs sind erfasst.\\
		\midrule
		Aufwandschätzung & 20 Stunden\\
		\midrule
		Priorität & Hoch\\
		\bottomrule
	\end{tabularx}
	\newpage

	\subsubsection{AP: Automatische Geräteerfassung}
	\begin{tabularx}{\textwidth}{lX}
		\toprule
		Beschreibung & Damit das Inventar nicht von Hand geführt werden muss, lassen sich die Geräteeinstellungen automatisch erfassen.\\
		\midrule
		Akzeptanzkriterien & Die Erfassung lässt sich erfolgreich durchführen, sie erfasst sämtliche Geräte im Netzwerksystem und speichert die Daten im YAML-Format.\\
		 & Unittests sind erstellt und laufen erfolgreich durch. Allfällige Bugs sind im Issue-Tracker erfasst.\\
		\midrule
		Aufwandschätzung & 30 Stunden\\
		\midrule
		Priorität & Niedrig\\
		\bottomrule
	\end{tabularx}
	\newpage

	\subsection{Kategorie Hilfsklassen}

	\subsubsection{AP: FileHandler}
	\begin{tabularx}{\textwidth}{lX}
		\toprule
		Beschreibung & Der Filehandler wird immer dann aufgerufen, wenn etwas aus dem Filesystem geladen werden muss oder etwas darin geschrieben werden soll.\\
		\midrule
		Akzeptanzkriterien & Der Filehandler ist implementiert und es lassen sich damit Dokumente im YAML Format schreiben und lesen.\\
		\midrule
		Aufwandschätzung & 6 Stunden\\
		\midrule
		Priorität & Hoch\\
		\bottomrule
	\end{tabularx}

	\subsubsection{AP: Logging}
	\begin{tabularx}{\textwidth}{lX}
		\toprule
		Beschreibung & Der Logger ist dafür zuständig, dass alle Events, die in der Software geschehen, beispielsweise Exceptions oder Informations-Logs, persistent gespeichert werden.\\
		\midrule
		Akzeptanzkriterien & Der Logger ist implementiert.\\
		\midrule
		Aufwandschätzung & 6 Stunden\\
		\midrule
		Priorität & Niedrig\\
		\bottomrule
	\end{tabularx}

	\subsubsection{AP: Integration Tests}
	\begin{tabularx}{\textwidth}{lX}
		\toprule
		Beschreibung & Mit den Integration Tests wird überprüft, ob die Softwarekomponenten alle miteinander zusammen agieren können. Sie dienen der Überprüfung des Systems.\\
		\midrule
		Akzeptanzkriterien & Integrationstests wurden systematisch und geplant durchgeführt und die Resultate dokumentiert. Allfällige Fehler wurden im Issue-Tracking erfasst.\\
		\midrule
		Aufwandschätzung & 10 Stunden\\
		\midrule
		Priorität & Niedrig\\
		\bottomrule
	\end{tabularx}
	
	\subsubsection{AP: Kommandozeilen-GUI}
	\begin{tabularx}{\textwidth}{lX}
		\toprule
		Beschreibung & Um mit dem Programm interagieren zu können, wird ein Grafikinterface benötigt.\\
		 & Vorerst wird dieses mit einem Kommandozeilen-GUI umgesetzt.\\
		\midrule
		Akzeptanzkriterien & Es ist möglich, die Erfassung, Orchestrierung und Durchführung über das GUI zu steuern.\\
		 & Usability-Tests wurden durchgeführt und daraus entstandene Verbesserungspunkte aufgenommen.\\
		\midrule
		Aufwandschätzung & 14 Stunden\\
		\midrule
		Priorität & Mittel \\
		\bottomrule
	\end{tabularx}
	\newpage

	

	\subsection{Kategorie Test-Integration}
	Die Testintegration befasst sich mit der Implementation der einzelnen Netzwerktests. 
	Für die meisten Tests sollte ein Aufwand von vier bis acht Stunden ausreichend sein, wobei umfangreichere Tests durchaus mehr Zeit für die Implementation benötigen können.
	Die Integration von weiteren Tests hat geringe Priorität, solange die Kernfunktionalität des Programms noch nicht fertiggestellt ist. 
	Ausnahme dafür ist der Ping-Test, welcher für die Entwicklung zum Ausprobieren verwendet wird und ein bis drei weitere Tests, welche für das Integration-Testing verwendet werden.

	Mögliche Tests:
	\begin{itemize}
		\item ping
		\item tracert
		\item show ip interface 
		\item show ip protocols
		\item show ip route
		\item show interfaces
		\item iperf
		\item nmap portscan
		\item show ip ospf interface
		\item show ip ospf neighbor
		\item show ip eigrp neighbors
		\item show tag-switching tdp discovery
		\item show tag-switching tdp bindings
		\item show tag-switching forwarding-table
		\item show ip bgp 
		\item show ip route bgp 
		\item show ip bgp neighbors
		\item show mpls lsp extensive
		\item show isis adjacency
		\item show isis interface
	\end{itemize}
\end{document}