\documentclass[
	ngerman,
	toc=listof, % Abbildungsverzeichnis sowie Tabellenverzeichnis in das Inhaltsverzeichnis aufnehmen
	toc=bibliography, % Literaturverzeichnis in das Inhaltsverzeichnis aufnehmen
	footnotes=multiple, % Trennen von direkt aufeinander folgenden Fußnoten
	parskip=half, % vertikalen Abstand zwischen Absätzen verwenden anstatt horizontale Einrückung von Folgeabsätzen
	numbers=noendperiod % Den letzten Punkt nach einer Nummerierung entfernen (nach DIN 5008)
]{scrartcl}
\pdfminorversion=5 % erlaubt das Einfügen von pdf-Dateien bis Version 1.7, ohne eine Fehlermeldung zu werfen (keine Garantie für fehlerfreies Einbetten!)

% Dokumenteninformationen ----------------------------------------------------
\newcommand{\titel}{Zeiterfassung}
\newcommand{\untertitel}{Studienarbeit \semester}
\newcommand{\kompletterTitel}{\titel{} \\ \untertitel}
\newcommand{\datum}{\today}

\newcommand{\vorlagenOrdner}{../99_Vorlagen} % Falls im Unterordner ../ vorne hinzufügen

\newcommand{\betriebLogo}{\vorlagenOrdner/Bilder/logo}

% Konfiguration -------------------------------------------------------------
\newcommand{\autoren}{
    \author{
        Schmid, Mike\\
        \texttt{sgschwin@hsr.ch}
        \and
        Schlatter, Janik\\
        \texttt{jschlatt@hsr.ch}
    }
}

\newcommand{\betreuer}{
    Stettler Beat\\
    \scriptsize \texttt{\url{beat.stettler@hsr.ch}}
    \normalsize
}

\newcommand{\schmid}{
    Mike Schmid\\
    \url{mschmid@hsr.ch}
    \normalsize
}

\newcommand{\schlatter}{
    Janik Schlatter\\
    \scriptsize \url{jschlatt@hsr.ch}
    \normalsize
}

\newcommand{\autorenNamen}{
    M. Schmid, J. Schlatter
}

\newcommand{\semester}{FS-2020}
\newcommand{\betriebName}{\textsc{HSR} Hochschule für Technik Rapperswil} % Metadaten zu diesem Dokument (Autor usw.)
\input{\vorlagenOrdner/Konfiguration/Packages} % verwendete Packages
\input{\vorlagenOrdner/Konfiguration/Seitenstil_Bericht} % Definitionen zum Aussehen der Seiten
\input{\vorlagenOrdner/Konfiguration/Befehle} % eigene allgemeine Befehle, die z.B. die Arbeit mit LaTeX erleichtern

\begin{document}

% Deckblatt ------------------------------------------------------------------
\phantomsection
\thispagestyle{plain}
\pdfbookmark[1]{Deckblatt}{deckblatt}
\begin{titlepage}
    \begin{center}
        \includegraphics[scale=1.5]{\betriebLogo}\\[10ex]

        \rule{\linewidth}{0.5mm}\\[2ex]
        {\huge \bfseries  \titel }\\[2ex]
        {\LARGE \untertitel }\\[2ex]
        {\large \datum}\\
        \rule{\linewidth}{0.5mm}\\[10ex]

        \begin{minipage}[t]{0.4\textwidth}
            \begin{flushleft} 
                \large \emph{Autoren:}\\
                    \large Mike \textsc{Schmid}\\
                    \scriptsize \texttt{mike.schmid@hsr.ch}\\[1ex]
                    \large Janik \textsc{Schlatter}\\
                    \scriptsize \texttt{janik.schlatter@hsr.ch}\\[1ex]
            \end{flushleft}
            \end{minipage}
            ~
            \begin{minipage}[t]{0.4\textwidth}
            \begin{flushright} 
                \large \emph{Supervisor:} \\
                Prof. Stettler \textsc{Beat}\\
                \scriptsize \texttt{beat.stettler@hsr.ch}\\[1ex]
            \end{flushright}
        \end{minipage}\\[40ex]

        \small
        \noindent
        Dieses Werk einschließlich seiner Teile ist \textbf{urheberrechtlich geschützt}.
        Jede Verwertung außerhalb der engen Grenzen des Urheberrechtgesetzes ist ohne
        Zustimmung des Autors unzulässig und strafbar. Das gilt insbesondere für
        Vervielfältigungen, Übersetzungen, Mikroverfilmungen sowie die Einspeicherung
        und Verarbeitung in elektronischen Systemen.

    \end{center}
\end{titlepage}
\cleardoublepage

% Preface --------------------------------------------------------------------
\pagenumbering{Roman}

% Zweck
\section*{Zweck}
Dieses Dokument beschreibt die Zeiterfassung von allen Projektmitgliedern.

% Inhaltsverzeichnis
\phantomsection
\pdfbookmark[1]{Inhaltsverzeichnis}{inhalt}
\tableofcontents
\cleardoublepage

\pagenumbering{arabic}
% Jede Überschrift 1 auf neuer Seite
\let\stdsection\section
\renewcommand\section{\clearpage\stdsection}

% Inhalt ---------------------------------------------------------------------
\section{Zeiterfassung Janik Schlatter}
	\begin{tabularx}{\textwidth}{llXl}
		\toprule
		Datum & Phase & Beschreibung & Stunden \\
		\midrule
		13.02.2020 & Init & Erstbesprechung M.Schmid \& J.Schlatter & 1.5 \\
		17.02.2020 & Init & Installation der notwendigen Tools & 1.5 \\
		17.02.2020 & Init & Einarbeiten in LateX \& Python & 5 \\
		18.02.2020 & Init & Aufetzen der Late Templates & 8 \\
		20.02.2020 & Init & Aufarbeiten Cn1 \ Cn2 & 3 \\
		20.02.2020 & Init & Meeting Woche 1 & 1 \\
		20.02.2020 & Init & Sitzungsprotokoll & 0.5 \\
		\midrule
		Zwischensumme: & 22.5 & & \\
		\midrule
		25.02.2020 & Analyse & Arbeiten am Projektplan & 7 \\
		27.02.2020 & Analyse & Meeting Woche 2 & 1 \\
		27.02.2020 & Analyse & Korrekturen gemäss Meeting & 1 \\
		27.02.2020 & Analyse & Sitzungsprotokoll & 0.5 \\
		27.02.2020 & Analyse & Arbeiten an der Anforderungsanalyse & 5 \\
		\midrule
		Zwischensumme: & 14.5 & & \\
		\midrule
		03.03.2020 & Analyse 2 & Recherche Nornir und Openconfig & 6 \\
		05.03.2020 & Analyse 2 & Recherche Netzwerktests und Protokolle & 3 \\
		05.03.2020 & Analyse 2 & Arbeiten an Use Cases und UseCaseDiagramm & 3 \\
		05.03.2020 & Analyse 2 & Meeting Woche 3 & 1 \\
		\midrule
		Zwischensumme: & 13 & & \\
		\midrule
		10.03.2020 & Analyse 3 & Arbeiten an Klassendiagramm & 4 \\
		10.03.2020 & Analyse 3 & Arbeiten an Systemsequenzdiagramm & 4 \\
		11.03.2020 & Analyse 3 & Meeting Woche 4 & 1 \\
		11.03.2020 & Analyse 3 & Nachbesprechung Meeting 4 & 2 \\
		12.03.2020 & Analyse 3 & Revision Anforderungsanalyse & 4 \\
		14.03.2020 & Analyse 3 & Sitzungsprotokoll Woche 4 & 0.5 \\
		\midrule
		Zwischensumme: & 15.5 & & \\
		\midrule
		17.03.2020 & Design 1 & Domainmodell & 7 \\
		18.03.2020 & Design 1 & Meeting Woche 4 & 1 \\
		19.03.2020 & Design 1 & Klassendiagramm & 8 \\
		\midrule
		Zwischensumme & 16 & & \\
		\midrule
		24.03.2020 & Design 2 & Einarbeiten Python & 8 \\
		25.03.2020 & Design 2 & Meeting Woche 5 & 1 \\
		26.03.2020 & Design 2 & Architekturendokument & 4 \\
		26.03.2020 & Design 2 & Arbeit am Prototyp & 3 \\
		\midrule
		Zwischensumme & 16 & & \\
		\bottomrule
	\end{tabularx}

\section{Zeiterfassung Mike Schmid}
	\begin{tabularx}{\textwidth}{llXl}
		\toprule
		Datum & Phase & Beschreibung & Stunden \\
		\midrule
		03.02.2020 & Init & Erstbesprechung mit B.Stettler und U.Baumann & 0.75 \\
		13.02.2020 & Init & Erstbesprechung M.Schmid \& J.Schlatter & 1.5 \\
		18.02.2020 & Init & Vorbereiten GitHub, Einarbeiten LaTeX, Recherche Restconf/Netconf, Aufarbeiten CN1\&2 & 6 \\
		19.02.2020 & Init & Repetition CN1\&2 & 2 \\
		20.02.2020 & Init & Einarbeitung und Sitzung & 4 \\
		21.02.2020 & Init & Recherche Nornir, Napalm, Openconfig & 1 \\
		\midrule
		Zwischensumme: & 15.25 & & \\
		\midrule
		25.02.2020 & Analyse 1 & Projektplan erstellen & 6 \\
		26.02.2020 & Analyse 1 & Anpassungen Projektplan & 3 \\
		27.02.2020 & Analyse 1 & Meeting Woche 2 & 1 \\
		27.02.2020 & Analyse 1 & Korrekturen gem. Meeting & 1 \\
		27.02.2020 & Analyse 1 & Nachtrag Arbeitspakete & 3 \\
		27.02.2020 & Analyse 1 & Arbeiten an Anforderungsanalyse & 2 \\
		\midrule
		Zwischensumme: & 16 & & \\
		\midrule
		03.03.2020 & Analyse 2 & Recherche Nornir und Openconfig & 6 \\
		05.03.2020 & Analyse 2 & Recherche Netzwerktests und Protokolle & 3 \\
		05.03.2020 & Analyse 2 & Arbeiten an Anforderungsanalyse & 4 \\
		05.03.2020 & Analyse 2 & Meeting Woche 3 & 1 \\
		\midrule
		Zwischensumme: & 14 & & \\
		\midrule
		10.03.2020 & Analyse 3 & Anforderungsanalyse Nonfunctional Requirements & 8 \\
		11.03.2020 & Analyse 3 & Meeting Woche 4 & 1 \\
		11.03.2020 & Analyse 3 & Nachbesprechung Meeting 4 & 2 \\
		12.03.2020 & Analyse 3 & Revision Anforderungsanalyse & 8 \\
		14.03.2020 & Analyse 3 & Revision Anforderungsanalyse & 4 \\
		\midrule
		Zwischensumme: & 23 & & \\
		\midrule
		17.03.2020 & Design 1 & Domainmodell und Prosa & 7 \\
		18.03.2020 & Design 1 & Meeting Woche 4 & 1 \\
		19.03.2020 & Design 1 & Klassendiagramm & 8 \\
		\midrule
		Zwischensumme & 16 & & \\
		\bottomrule
	\end{tabularx}
	
	\newpage

	\begin{tabularx}{\textwidth}{llXl}
		\toprule
		Datum & Phase & Beschreibung & Stunden \\
		\midrule
		24.03.2020 & Design 2 & Einarbeiten Python & 8 \\
		25.03.2020 & Design 2 & Meeting Woche 5 & 1 \\
		26.03.2020 & Design 2 & Architekturentscheidungen & 4 \\
		26.03.2020 & Design 2 & Arbeit am Prototyp & 3 \\
		\midrule
		Zwischensumme & 16 & & \\
		\midrule
		31.03.2020 & Design 3 & Architekturentscheidungen & 3 \\
		31.03.2020 & Design 3 & Arbeitspakete erweitern &  5 \\
		01.03.2020 & Design 3 & Meeting Woche 6 & 1 \\
		02.03.2020 & Design 3 & Revision Architekturentscheidungen & 5 \\
		\midrule
		Total & 106.25 & (Soll: 240) & \\
		\bottomrule
	\end{tabularx}

\end{document}