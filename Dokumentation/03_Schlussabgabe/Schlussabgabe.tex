\documentclass[
	ngerman,
	toc=listof, % Abbildungsverzeichnis sowie Tabellenverzeichnis in das Inhaltsverzeichnis aufnehmen
	toc=bibliography, % Literaturverzeichnis in das Inhaltsverzeichnis aufnehmen
	footnotes=multiple, % Trennen von direkt aufeinander folgenden Fußnoten
	parskip=half, % vertikalen Abstand zwischen Absätzen verwenden anstatt horizontale Einrückung von Folgeabsätzen
	numbers=noendperiod % Den letzten Punkt nach einer Nummerierung entfernen (nach DIN 5008)
]{scrartcl}
\pdfminorversion=5 % erlaubt das Einfügen von pdf-Dateien bis Version 1.7, ohne eine Fehlermeldung zu werfen (keine Garantie für fehlerfreies Einbetten!)

% Dokumenteninformationen ----------------------------------------------------
\newcommand{\titel}{Network Unit Testing System}
\newcommand{\untertitel}{Studienarbeit \semester}
\newcommand{\kompletterTitel}{\titel{} \\ \untertitel}
\newcommand{\datum}{\today}

\newcommand{\vorlagenOrdner}{../99_Vorlagen} % Falls im Unterordner ../ vorne hinzufügen

\newcommand{\betriebLogo}{\vorlagenOrdner/Bilder/logo}

% Konfiguration -------------------------------------------------------------
\newcommand{\autoren}{
    \author{
        Schmid, Mike\\
        \texttt{sgschwin@hsr.ch}
        \and
        Schlatter, Janik\\
        \texttt{jschlatt@hsr.ch}
    }
}

\newcommand{\betreuer}{
    Stettler Beat\\
    \scriptsize \texttt{\url{beat.stettler@hsr.ch}}
    \normalsize
}

\newcommand{\schmid}{
    Mike Schmid\\
    \url{mschmid@hsr.ch}
    \normalsize
}

\newcommand{\schlatter}{
    Janik Schlatter\\
    \scriptsize \url{jschlatt@hsr.ch}
    \normalsize
}

\newcommand{\autorenNamen}{
    M. Schmid, J. Schlatter
}

\newcommand{\semester}{FS-2020}
\newcommand{\betriebName}{\textsc{HSR} Hochschule für Technik Rapperswil} % Metadaten zu diesem Dokument (Autor usw.)
\input{\vorlagenOrdner/Konfiguration/Packages} % verwendete Packages
\input{\vorlagenOrdner/Konfiguration/Seitenstil_Bericht} % Definitionen zum Aussehen der Seiten
\input{\vorlagenOrdner/Konfiguration/Befehle} % eigene allgemeine Befehle, die z.B. die Arbeit mit LaTeX erleichtern

\begin{document}

% Deckblatt ------------------------------------------------------------------
\phantomsection
\thispagestyle{plain}
\pdfbookmark[1]{Deckblatt}{deckblatt}
\begin{titlepage}
    \begin{center}
        \includegraphics[scale=1.5]{\betriebLogo}\\[10ex]

        \rule{\linewidth}{0.5mm}\\[2ex]
        {\huge \bfseries  \titel }\\[2ex]
        {\LARGE \untertitel }\\[2ex]
        {\large \datum}\\
        \rule{\linewidth}{0.5mm}\\[10ex]

        \begin{minipage}[t]{0.4\textwidth}
            \begin{flushleft} 
                \large \emph{Autoren:}\\
                    \large Mike \textsc{Schmid}\\
                    \scriptsize \texttt{mike.schmid@hsr.ch}\\[1ex]
                    \large Janik \textsc{Schlatter}\\
                    \scriptsize \texttt{janik.schlatter@hsr.ch}\\[1ex]
            \end{flushleft}
            \end{minipage}
            ~
            \begin{minipage}[t]{0.4\textwidth}
            \begin{flushright} 
                \large \emph{Supervisor:} \\
                Prof. Stettler \textsc{Beat}\\
                \scriptsize \texttt{beat.stettler@hsr.ch}\\[1ex]
            \end{flushright}
        \end{minipage}\\[40ex]

        \small
        \noindent
        Dieses Werk einschließlich seiner Teile ist \textbf{urheberrechtlich geschützt}.
        Jede Verwertung außerhalb der engen Grenzen des Urheberrechtgesetzes ist ohne
        Zustimmung des Autors unzulässig und strafbar. Das gilt insbesondere für
        Vervielfältigungen, Übersetzungen, Mikroverfilmungen sowie die Einspeicherung
        und Verarbeitung in elektronischen Systemen.

    \end{center}
\end{titlepage}
\cleardoublepage

% Preface --------------------------------------------------------------------
\pagenumbering{Roman}


\addcontentsline{toc}{section}{Aufgabenstellung}
\subfile{Kapitel/Aufgabenstellung}

\newpage

\subfile{Kapitel/Abstract}
\setcounter{page}{2}
\addcontentsline{toc}{section}{Abstract}

\newpage


\subfile{Kapitel/ManagementSummary}
\addcontentsline{toc}{section}{Management Summary}
\cleardoublepage

% Inhaltsverzeichnis
\phantomsection
\pdfbookmark[1]{Inhaltsverzeichnis}{inhalt}
\tableofcontents
\cleardoublepage

\pagenumbering{arabic}
% Jede Überschrift 1 auf neuer Seite
\let\stdsection\section
\section*{I. Technischer Bericht}
\addcontentsline{toc}{section}{I. Technischer Bericht}

% Inhalt ---------------------------------------------------------------------
\subfile{Kapitel/Einleitung.tex}
\renewcommand\section{\clearpage\stdsection}
\subfile{Kapitel/Anforderungen.tex}
\subfile{Kapitel/Design.tex}
\subfile{Kapitel/Realisierung.tex}
\subfile{Kapitel/Ergebnisse.tex}
\subfile{Kapitel/Ausblick.tex}
\subfile{Kapitel/Glossar.tex}

% Anhang----------------------------------------------------------------------

\section*{II. Anhang}
\renewcommand\section{\stdsection}
\setcounter{page}{1}
\setcounter{section}{0}
\addcontentsline{toc}{section}{II. Anhang}
\subfile{Kapitel/Anhang/Projektplanung.tex}
\renewcommand\section{\clearpage\stdsection}
\subfile{Kapitel/Anhang/Zeitauswertung.tex}
\subfile{Kapitel/Anhang/Benutzeranleitung.tex}
\subfile{Kapitel/Anhang/Persönliche Berichte.tex}
\subfile{Kapitel/Anhang/Eigenständigkeitserklärung.tex}
\subfile{Kapitel/Anhang/Rechtliches.tex}

\end{document}