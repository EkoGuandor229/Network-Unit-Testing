\documentclass[]{subfiles}

\begin{document}
    \section{Einleitung}
    \subsection{Problemstellung}
    Netzwerke bestehen aus dutzenden bis tausenden Komponenten. 
    Jede dieser Komponente hat eine eigene Konfiguration und Aufgabe.
    In den letzten Jahren ist es durch die softwaregesteuerte Konfiguration
    von Netzwerkgeräten einfacher geworden, die Komponenten für ihre
    Aufgabe einzustellen. Trotzdem werden die Überprüfungen der Konfiguration 
    auch heute noch manuell vorgenommen. Dies kann dazu führen, dass wegen menschlicher
    Fehler ein Fehlverhalten eines der Netzwerkkomponente dazu führt,
    dass das gesammte Netzwerk gestört wird. 
    Weiterhin wird die Überprüfung noch komplizierter, da neben der statischen 
    Konfiguration sich das Netzwerk zur Laufzeit dynamisch anpasst, 
    um die Performanz des Systems zu optimieren und Fehler sowie Ausfälle
    zu korrigieren. 
    Dieses Verhalten wird über verschiedene Netzwerkprotokolle gesteuert, z.B. OSPF oder
    BGP.
    
    In der Softwareentwicklung werden schon seit 
    Ende der 80er-Jahre Komponententests, sogenannte Unit-Tests durchgeführt, um einzelne 
    Komponenten (Units) automatisiert zu Testen. Dabei wird ein Computerprogramm ausgeführt,
    welches mit verschiedenen Eingabeparametern überprüft, ob die Ausgabe des zu testenden
    Programms den erwarteten Ergebnissen entspricht.

    Dabei ist es möglich die Tests vor und nach einer geplanten Änderung durchzuführen,
    um zu überprüfen, ob die Software innerhalb der definierten Funktionsparameter operiert.
    Tests sollen dafür so geschrieben werden, dass möglichst jede Situation mit den 
    Eingabeparametern abgebildet wird. Unittests sollen regelmässig durchgeführt werden,
    damit Fehler früh gefunden werden und sich nicht auf das gesammte System auswirken können.

    \newpage
    
    \subsection{Aufgabenstellung}
    \label{subsec:Aufgabenstellung}
    Ziel dieser Arbeit ist, ein Programm zu entwickeln, mit dem sich Netzwerktests mit 
    der gleichen Arbeitsweise durchführen lassen, wie Unittests in der Softwareentwicklung
    durchgeführt werden. Dabei müssen folgende Anforderungen an die Tests erfüllt sein:
    \begin{itemize}
        \item Tests müssen planbar sein. Es soll ein Testplan existieren.
        \item Tests müssen systematisch spezifiziert werden. Es existieren Test-Spezifikationen.
        \item Testresultate werden Dokumentiert.
        \item Tests sollen, wo möglich, automatisiert durchgeführt werden.
        \item Testergebnisse müssen reproduzierbar und nachvollziehbar sein.
    \end{itemize}
    Es soll evaluiert werden, welche bereits verfügbaren Tools sich für ein solches Testprogramm
    eignen. Die Umsetzung soll diese Tools einbinden.
    Eine Wichtige Anforderung an das zu entwickelnde System ist, dass sich weitere 
    Netzwerktests möglichst einfach hinzufügen lassen, ohne dass dafür der gesamte Code
    geändert werden muss. 
    \end{document}

   \subsection{Herausforderungen}
   Eine der grössten Herausforderungen an ein automatisiertes Testsystem sind die 
   verschiedenen Protokolle und die Unterschiede der Standards von diversen Herstellern.

   Ohne Kenntnisse, welche Protokolle auf einem Netzwerkgerät konfiguriert sind, 
   können diese Netzwerkgeräte nicht effizient getestet werden, da die Netzwerkprotokolle
   das Verhalten der Geräte zur Laufzeit beeinflussen. 
   Deshalb müssen für die Testdurchführung im vornherein die Konfigurationen und verwendeten
   Protokolle der Netzwerkgeräte bekannt sein und ein Testsystem muss mit diesen interagieren
   können.

   Unterschiedliche Hersteller haben verschiedene Kommandozeilenbefehle (CLI-Commands) 
   für ihre Geräte, welche für die Konfiguration und Abfrage der Konfiguration verwendet
   werden.
   Ausserdem kann es vorkommen, dass ein Hersteller mit der Einführung einer neuen 
   Version des Gerätebetriebssystems neue CLI-Commands einführt oder alte Befehle
   ändert.
   Dies setzt eine enorme flexibilität für ein Testprogramm voraus, welches ein 
   beliebiges Netzwerk mit unterschiedlichen Geräten von verschiedenen Herstellern 
   testen soll. 

   \subsection{Vorarbeit}
   Die Studienarbeit 'Network Unit Testing' aus dem Herbstsemester 2016 von 
   Andreas Stalder und David Meister hat sich bereits mit dem Thema automatisierte 
   Netzwerktests auseinandergesetzt. 
   Der Fokus lag dabei auf dem Ausarbeiten einer Testdefinitionssprache, mit der 
   solche Netzwerktests annotiert werden können. 
   Zudem wurde mit SaltStack ein Konfigurationsmanagement-Tool evaluiert und auf 
   dessen basis eine Software entwickelt, mit der man Netzwerktests ausführen konnte.

   Die grössten Herausforderungen der Vorarbeit lagen dabei auf den Unterschieden 
   verschiedener Hersteller und darin, dass die Outputformate der Neztwerktests 
   je nach Hersteller anders zu Parsen sind. 
   Diese Herausforderungen haben dazu geführt, dass beim Parsen der Ergebnisse 
   oftmals auf reguläre Ausdrücke zurückgegriffen wurde, deren Implementation
   umständlich und schwierig zu lesen ist.
   Die Autoren haben dabei die Hoffnung angemerkt, dass man die Resultate künftig
   einfacher verarbeiten kann und die Hersteller ein einheitliches Rückgabeformat 
   verwenden.

   Der grösste Nachteil in der Verwendung der Lösung aus der Vorarbeit besteht darin, 
   dass die Implementation neuer Netzwerktests eher umständlich ist.
   Ausserdem besteht eine starke Abhängigkeit zum Tool SaltStack, bei dem 
   keine Garantie für eine langfristige Verfügbarkeit besteht.

   Zukünftige Arbeiten sollen desshalb eine geringere Abhängigkeit zu externen Tools
   haben und die erweiterung um neue Tests so einfach wie möglich umsetzen.

