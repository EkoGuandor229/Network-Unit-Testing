\documentclass[]{subfiles}

\begin{document}
    \section{Einleitung}
    \subsection{Problemstellung}
    Netzwerke bestehen aus dutzenden bis tausenden Komponenten. 
    Jede dieser Komponente hat eine eigene Konfiguration und Aufgabe.
    In den letzten Jahren ist es durch die softwaregesteuerte Konfiguration
    von Netzwerkgeräten einfacher geworden, die Komponenten für ihre
    Aufgabe einzustellen. Trotzdem werden die Überprüfungen der Konfiguration 
    auch heute noch manuell vorgenommen. Dies kann dazu führen, dass wegen menschlicher
    Fehler ein Fehlverhalten eines der Netzwerkkomponente dazu führt,
    dass das gesammte Netzwerk gestört wird. 
    
    In der Softwareentwicklung werden schon seit 
    Ende der 80er-Jahre Komponententests, sogenannte Unit-Tests, durchgeführt, um einzelne 
    Komponenten (Units) automatisiert zu Testen. Dabei wird ein Computerprogramm ausgeführt,
    welches mit verschiedenen Eingabeparametern überprüft, ob die Ausgabe des zu testenden
    Programms den erwarteten Ergebnissen entspricht.

    Dabei ist es möglich die Tests vor und nach einer geplanten Änderung durchzuführen,
    um zu überprüfen, ob die Software innerhalb der definierten Funktionsparameter operiert.
    Tests sollen dafür so geschrieben werden, dass möglichst jede Situation mit den 
    Eingabeparametern abgebildet wird und Unittests sollen regelmässig durchgeführt werden,
    damit Fehler früh gefunden werden und sich nicht auf das gesammte System auswirken können.

    \subsection{Aufgabenstellung}
    Ziel dieser Arbeit ist, ein Programm zu entwickeln, mit dem sich Netzwerktests mit 
    der gleichen Arbeitsweise durchführen lassen, wie Unittests in der Softwareentwicklung
    durchgeführt werden. Dabei müssen folgende Punkte erfüllt sein:
    \begin{itemize}
        \item Tests müssen planbar sein. Es soll ein Testplan existieren.
        \item Tests müssen systematisch spezifiziert werden. Es existieren Test-Spezifikationen.
        \item Testresultate werden Dokumentiert.
        \item Tests sollen, wo möglich, automatisiert durchgeführt werden.
        \item Testergebnisse müssen reproduzierbar und nachvollziehbar sein.
    \end{itemize}
    Es soll evaluiert werden, welche bereits verfügbaren Tools sich für ein solches Testprogramm
    eignen. Die Umsetzung soll diese Tools einbinden. 
    \end{document}