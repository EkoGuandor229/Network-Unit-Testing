\documentclass[]{subfiles}

\begin{document}
\section*{Management Summary}
   \subsection*{Ausgangslage}
   Fehler in Teilbereichen von Netzwerksystemen können dazu führen, 
   dass das ganze System nicht mehr funktioniert.
   Aus diesem Grund ist es essenziell, dass selbst kleine Änderungen
   an Netzwerken getestet werden können. 
   Diese Tests werden meistens von Hand oder durch Skripte durchgeführt.
   Ein Tool, welches das automatisierte Testen von Netzwerksystemen 
   ermöglicht, kann dabei helfen, Fehler zu erkennen, bevor sie zu
   einem Problem werden.

   \subsection*{Vorgehen, Technologien}
   Zu Beginn wurde eine Domänenanalyse durchgeführt, um die Akteure und 
   Bestandteile einer Netzwerkumgebung zu bestimmen und die zu entwickelnden
   Netzwerktests zu evaluieren.
   Darauf aufbauend wurden die funktionalen und nichtfunktionalen Anforderungen
   an die Software spezifiziert.
   Auf dieser Basis wurde die Softwarearchitekur ausgearbeitet und mit 
   der Entwicklung begonnen.

   Das Programm wurde in der Programmiersprache Python geschrieben und 
   beinhaltet das Modul "Nornir"\,, ein Framework, welches automatisierte
   Tasks auf Netzwerksysteme, wie z.B. Konfiguration oder Tests, ermöglicht. 

   \subsection*{Ergebnisse}
   Aus dieser Studienarbeit ist ein Python-Programm entstanden, welches 
   Netzwerktests, die in einer Definitionssprache spezifiziert werden,
   gegen ein Netzwerk durchführt und die Ergebnisse selbstständig auswertet
   und dem Benutzer anzeigt. Nornir erlaubt dabei, eine Vielzahl von 
   Netzwerkgräten anzusprechen, welche über herkömmliche Methoden wie
   SSH umfänglicher zu testen wären.

   Die Software lässt sich ohne Installation auf jedem Gerät ausführen, 
   welches Python-Code ausführen kann, unabhängig vom Betriebssystem.
   Geräte, auf denen Python nicht installiert ist, müssen dies zuerst
   Installieren, können das Programm danach aber ohne weiteres ausführen.

   Dadurch, dass das Programm reiner Pyhton Code ist, lässt es sich einfach
   in ein bestehendes Tool für die kontinuierliche Integration einbinden.
   Die Testdefinitionen lassen sich über ein Versionsverwaltungstool 
   zentralisieren, so dass mehrere Netzwerkleute gleichzeitig Tests für 
   eine Umgebung entwickeln können. 
   
\end{document}