\documentclass[]{subfiles}

\begin{document}
\section*{Aufgabenstellung}

Änderungen an Netzwerkumgebungen werden in der Praxis auch heute 
noch durch Kommandozeilenbefehle oder Skripte getestet.
Diese Tests beinhalten oft einfache Befehle wie 'ping' oder 'traceroute'.
Im Vergleich dazu werden Softwareprojekte durch automatisierte Tests, 
welche regelmässig ausgeführt werden, getestet.
Sogenannte Unit Tests werden vor und nach einer Änderung durchgeführt, 
um zu Testen, ob sich ein Programm weiterhin innerhalb der geforderten 
Betriebsparameter verhält. 
Somit können Fehler schnell gefunden und behoben werden und die Robustheit
der Software wird erhöht.
Ein vergleichbarer Arbeitsablauf soll auch für den Netzwerkbereich ermöglicht werden.

Eine frühere Studienarbeit hat sich mit der Entwicklung einer Beschreibungssprache
befasst, mit derer solche Tests möglich wären.
Die Vorarbeit wurde dabei so entwickelt, dass das Programm mit dem Automationsframework 
SaltStack ausgeführt wurde. 
Diese Arbeit soll ein Programm entwickeln, welches selbstständig und unabhängig 
von anderen Programmen arbeiten kann.

Die Studierenden erhalten die Aufgabe, ein Python-Programm zu entwickeln, 
welches automatisierte Tests auf ein Netzwerk durchführen kann.
Das Programm soll gemäss einer Testdefinition selbstständig die 
auszuführenden Tests erstellen, durchführen und die Testresultate mit
einem Erwartungswert vergleichen. 
Die Auswertung der Tests soll direkt bei der Ausführung auf der Konsole
angezeigt werden und zusätzlich in einem Testreport für die spätere 
Ansicht gespeichert werden.
Die Programmausführung kann manuell oder automatisch über einen Deployment-Prozess
gestartet werden.

\vspace{5cm}

\begin{tabularx}{\textwidth}{XX}
    Prof.Beat Stettler & Urs Baumann \\
\end{tabularx}

    

\end{document}