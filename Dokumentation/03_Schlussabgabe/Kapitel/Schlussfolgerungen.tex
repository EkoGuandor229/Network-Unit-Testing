\documentclass[]{subfiles}

\begin{document}
\section{Schlussfolgerungen}
\subsection{Künftige Arbeiten}
\subsubsection*{Automatische Geräteerkennung}
Im aktuellen Programm müssen sämtliche Netzwerkgeräte von Hand in das Inventar aufgenommen
werden, was genaue Kenntnisse des Netzwerks voraussetzt und zeitintensiv ist.
Eine mögliche Erweiterung von NUTS2.0 wäre eine automatische Erkennung aller Geräte im 
Netzwerk und das automatische Erstellen des Inventars.
Dieses Vorgehen würde dem Anwender einiges an Arbeit einsparen und den Prozess für die 
Erstellung von Netzwerktests beschleunigen.

\subsubsection*{Anbindung einer Datenbank}
Die Anbindung einer Datenbank würde viele Vorteile mit sich bringen.
Die Verwaltung des Inventars oder der Testdefinitionen müsste nicht mehr in YAML-Files 
gemacht werden, sondern könnte mit einer Datenbank vereinfacht werden.
Das Speichern von Testresultaten in einer Datenbank würde es ermöglichen, 
spezifische Queries für individuelle Testresultate oder Testdurchführungen zu verwenden.
Dadurch könnte man wirkungsvolle Abfragen durchführen, um beispielsweise sämtliche 
Tests vor einer geplanten Änderung und danach abzurufen und auszuwerten.

\subsubsection*{Komplette GUI geführte Durchführung}
Die aktuelle Software hat nur ein minimalistisches GUI für die Auswahl und Reihenfolge
der Testdurchführung. 
Eine mögliche Erweiterung ist das Einbinden eines GUI für die gesammte Testdurchführung.
Ein Benutzer könnte z.B. die Testdefinitionen über ein Interface gesteuert erfassen, 
statt diese in einem YAML selber zu erstellen. 
Man könnte für sämtliche Testcommands beschreibungen anzeigen, was diese Tests machen
und wie sie verwendet werden. 
Es könnte eine Grafische Abbildung des Netzwerks angezeigt werden (Digitaler Zwilling), 
in dem man sämtliche Neztwerkgeräte mit ihren Parametern anzeigt. 

\subsubsection*{Erweiterung um Tests}
Zum Zeitpunkt der Studienarbeit wurden nur einige Netzwerktests implementiert, die 
sich auf das Testnetzwerk auch ausführen liessen. 
Das System liesse sich um sämtliche Netzwerktests erweitern, die in allen möglichen 
Netzwerken auch tatsächlich ausgeführt werden könnten.
Ausserdem gibt es Netzwerktests, die sich mit Nornir zum jetzigen Zeitpunkt gar nicht
ausführen lassen, z.B. die Grafische darstellung des OSPF spanning Tree. 
Solche Tests müssten manuell implementiert werden und benötigen allenfalls andere 
Module als Nornir.

\subsubsection*{Asynchrone Durchführung der Tests}
Momentan werden alle Netzwerktests nacheinander, d.H. Synchron durchgeführt.
Dadurch benötigt das Programm für die Durchführung von umfangreichen Testdefinitionen 
viel Zeit.
Man könnte in einer zukünftigen Erweiterung Tests mit der gleichen Kategorie, z.B.
Ping-Tests asynchron (paralell) ausführen, was zu einer kürzeren Durchführungszeit 
führen würde. 
\end{document}