\documentclass[]{subfiles}

\begin{document}
\section{Schlussfolgerungen}
\subsection{Endresultat}
Das Endergebnis der Studienarbeit hat leichte Abweichungen von dem, was wir zu Beginn
des Projekts geplant haben. 
In der Architektur wurde zuerst eine weitere Strategy-Factory für die Erstellung 
der Netzwerk-Verbindungen geplant. Diese wurde aber nicht implementiert, da 
sich herausgestellt hat, dass eine Strategy-Factory nur für weitere Komplexität 
im Code sorgt. 
Stattdessen wurde ein Mapping der Hostbetriebssysteme auf die 
kompatiblen Verbindungsschnittstellen mittels einem Dictionary vorgenommen.
Dieses Vorgehen bietet eine ähnliche Funktionalität, ist aber weitaus simpler 
in der Anwendung und im Verständnis.

Weiterhin wurde ein grosser Teil der Testdurchführungslogik in die konkreten Testimplementationen
verschoben. 
Am Anfang wurde geplant, dass jede Komponente die Logik beinhaltet die für die Testdurchführung,
die Evaluation der Ergebnisse und das Mapping der Resultatwerte benötigt werden.
Stattdessen wurde die Funktionalität des Mappers, des Evaluators und des Testrunners soweit 
vereinfacht, dass darin für jeden Test die jeweilige Methode aufgerufen wird, die diese
Logik spezifisch für jeden Test implementiert.
Durch dieses Vorgehen konnte beispielsweise das Mapping der Testresultate für jeden 
Netzwerktest innerhalb des Tests vorgenommen werden. 
Dadurch wurde viel Code im Mapper gespart, da jeder Test eine andere Formatierung der 
Rückgabewerte hat und dementsprechend für jeden Test ein spezifisches Mapping der Ergebnisse
hätte gemacht werden müssen.

Das Nornir-Framework erlaubt es dem Benutzer, schnell und einfach Netzwerkbefehle an ein
definiertes Netzwerkgerät zu senden und die Rückgabewerte werden dabei in einem 
Dictionary von Resultatwerten zurückgegeben. 
Dadurch müssen die Resultate nicht mit Verwendung von regulären Ausdrücken mühsam 
geparst werden, sondern sie können unter Anwendung von einfachen Operationen 
in eine einheitliche Form gebracht und mit den Erwartungswerten verglichen werden.
Der grösste Nachteil in der Verwendung von Nornir ist die Navigierung durch die Dokumentation.
Will man die einzelnen Netzwerktreiber (Napalm, Netmiko, etc.) verwenden, muss man 
oftmals in den Dokumentationen der Treiber selbst nachsehen, wie genau man die
Implementation nun vornehmen muss, vorausgesetzt es existiert eine spezifische Anleitung
für die Implementation. 

\newpage

\subsection{Vergleich mit der Vorarbeit}
Die Studienarbeit aus dem Herbstsemester 2016 hat im Vergleich mit dieser Studienarbeit
mehr konkrete Netzwerktests umgesetzt. 
Allerdings hat diese Studienarbeit einen höheren Fokus auf der Erweiterbarkeit um neue
Tests und es werden mehr verschiedene Gerätehersteller unterstützt.

Die reine Pyhton-Implementation erlaubt ein Deployment auf eine Vielzahl von Geräten 
mit unterschiedlichen Betriebssystemen unabhängig von externen Frameworks.
NUTS2.0 lässt sich in ein bestehendes Configuration Mangement Tool einbinden und kann
auf jedem Gerät ausgeführt werden, welches Python3 unterstützt.
Im Vergleich dazu wurde NUTS1.0 hauptsächlich für die Verwendung auf Linux-Systemen entwickelt.

Die Erweiterung des Systems ist in NUTS2.0 einfacher gestaltet, da sämtlicher Code in 
Python geschrieben ist und die verwendeten Module ebenfalls in Python entwickelt wurden.

\newpage




\subsection{Künftige Arbeiten}
\label{subsec:Künftige Arbeiten}
\subsubsection*{Automatische Geräteerkennung}
Im aktuellen Programm müssen sämtliche Netzwerkgeräte von Hand in das Inventar aufgenommen
werden, was genaue Kenntnisse des Netzwerks voraussetzt und zeitintensiv ist.
Eine mögliche Erweiterung von NUTS2.0 wäre eine automatische Erkennung aller Geräte im 
Netzwerk und das automatische Erstellen des Inventars.
Dieses Vorgehen würde dem Anwender einiges an Arbeit einsparen und den Prozess für die 
Erstellung von Netzwerktests beschleunigen.

\subsubsection*{Anbindung einer Datenbank}
Die Anbindung einer Datenbank würde viele Vorteile mit sich bringen.
Die Verwaltung des Inventars oder der Testdefinitionen müsste nicht mehr in YAML-Files 
gemacht werden, sondern könnte mit einer Datenbank vereinfacht werden.
Das Speichern von Testresultaten in einer Datenbank würde es ermöglichen, 
spezifische Queries für individuelle Testresultate oder Testdurchführungen zu verwenden.
Dadurch könnte man wirkungsvolle Abfragen durchführen, um beispielsweise sämtliche 
Tests vor einer geplanten Änderung und danach abzurufen und auszuwerten.

\subsubsection*{Komplette GUI geführte Durchführung}
Die aktuelle Software hat nur ein minimalistisches GUI für die Auswahl und Reihenfolge
der Testdurchführung. 
Eine mögliche Erweiterung ist das Einbinden eines GUI für die gesamte Testdurchführung.
Ein Benutzer könnte z.B. die Testdefinitionen über ein Interface gesteuert erfassen, 
statt diese in einem YAML selber zu erstellen. 
Man könnte für sämtliche Testcommands Beschreibungen anzeigen, was diese Tests machen
und wie sie verwendet werden. 
Es könnte eine Grafische Abbildung des Netzwerks angezeigt werden (Digitaler Zwilling), 
in dem man sämtliche Neztwerkgeräte mit ihren Parametern anzeigt. 

\subsubsection*{Erweiterung um Tests}
Zum Zeitpunkt der Studienarbeit wurden nur einige Netzwerktests implementiert, die 
sich auf das Testnetzwerk auch ausführen liessen. 
Das System liesse sich um sämtliche Netzwerktests erweitern, die in allen möglichen 
Netzwerken auch tatsächlich ausgeführt werden könnten.
Ausserdem gibt es Netzwerktests, die sich mit Nornir zum jetzigen Zeitpunkt gar nicht
ausführen lassen, z.B. die Grafische Darstellung des OSPF spanning Tree. 
Solche Tests müssten manuell implementiert werden und benötigen allenfalls andere 
Module als Nornir.

\paragraph{Mögliche Tests}
Die folgende Auflistung zeigt einige der gängigsten CLI-Befehle, wie sie im Netzwerkumfeld
verwendet werden um die Netzwerkumgebung zu testen. 

\begin{itemize}
    \item 'show running-configuration'
    \item 'show startup-configuration'
    \item 'show ip protocols'
    \item 'show ip route'
    \item 'show interfaces status'
    \item 'show interfaces trunk'
    \item 'show interfaces'
    \item 'show ip ospf interface'
    \item 'show ip ospf border-routers'
    \item 'show ip ospf virtual-links'
    \item 'show ip bgp'
    \item 'show ip bgp summary'
    \item 'show ip bgp neighbor'
    \item 'show ip bgp paths'
\end{itemize}


\subsubsection*{Asynchrone Durchführung der Tests}
Momentan werden alle Netzwerktests nacheinander, d.H. Synchron durchgeführt.
Dadurch benötigt das Programm für die Durchführung von umfangreichen Testdefinitionen 
viel Zeit.
Man könnte in einer zukünftigen Erweiterung Tests mit der gleichen Kategorie, z.B.
Ping-Tests asynchron (parallel) ausführen, was zu einer kürzeren Durchführungszeit 
führen würde. 
\end{document}

\subsubsection*{Automatische Erstellung repetitiver Tests}
Eine weitere Möglichkeit für die Erweiterung der Software ist, dass repetitive Netzwerktests
automatisch erstellt werden. Beispielsweise könnten Ping Tests für sämtliche Geräte, die im 
Inventory definiert sind vom Programm erstellt und nicht manuell erfasst werden.
Weiterhin könnte eine Erweiterung vorgenommen werden, die bestimmte Tests bei der Erfassung
vorschlägt, z.B. show ip interface brief für einen Router.