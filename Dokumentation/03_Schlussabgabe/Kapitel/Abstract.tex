\documentclass[]{subfiles}

\begin{document}
\section*{Abstract}
    Das testen von Netzwerkkonfigurationen findet auch heute noch hauptsächlich 
    mit handgeschriebenen CLI-Befehlen oder kleinen Skripten statt. 
    Wenn der Netzwerktechniker einen Fehler bei der Konfiguration macht, 
    oder etwas vergisst, kann es vorkommen, dass im Netzwerk Fehler auftreten, 
    deren Ursprung schwierig zu ermitteln ist und eine komplette Repetition der 
    (handgeschriebenen) Tests erfordert.
    Ein Programm, welches, wie in der Softwareentwicklung, vordefinierte und automatisch 
    durchgeführte Tests, sogenannte Unit-Tests, ermöglicht, könnte diese Probleme stark
    verringern. Dabei können zwei grobe Arbeitsvorgänge beschrieben werden. 
    Im ersten schreibt ein Netzwerktechniker Tests, die ein bestehendes Netzwerk 
    möglichst genau abbilden/beschreiben sollen. 
    Die Tests lassen sich jederzeit durchführen und testen den Zustand und die 
    Konfiguration des Netzwerks. 
    Falls nun ein Fehler auftritt, können die Tests automatisiert durchgeführt 
    werden und dann, vorausgesetzt sie sind vollständig, sollte der Report aufzeigen, 
    was genau schiefgegangen ist und wo der Fehler liegt. 
    Der Zweite Arbeitsvorgang entspricht dem in der Softwareentwicklung gängigen 
    Test-Driven-Development (TDD). 
    Beim TDD werden Tests geschrieben, bevor das System verändert wird, oder bevor 
    man neuen Code schreibt. 
    Auf ein Netzwerk abstrahiert könnte beispielsweise ein Administrator, der 
    eine Änderung am Netzwerk vornehmen will, zuerst die Tests schreiben, 
    welche die Änderung testen sollen. 
    Danach werden die Konfigurationen verändert und die Tests durchgeführt. 
    Falls die Tests nun fehlschlagen, kann man die Konfiguration anpassen oder 
    sogar auf einen früheren Zustand zurücksetzen. 
    Beide Arbeitsvorgänge erleichtern die Fehlersuche und erhöhen die Stabilität des Netzwerks.

    Aus dieser Arbeit ist das Programm "Nuts2.0"\, hervorgegangen, die vordefinierte
    Netzwerktests mit dem Automatisierungsframework Nornir durchführt und die 
    Ergebnisse ausgewertet darstellt. 
    Nornir ermöglicht es, dass unterschiedliche Geräte von verschiedenen Herstellern
    über mehrere Kommunikationskanäle angesprochen werden können und die Testresultate
    in einer einheitlichen Formatierung zurückgegeben werden.

\end{document}