\documentclass[]{subfiles}

\begin{document}
\section*{Abstract}
    Die Durchführung von Tests wird im Netzwerkbereich auch heute noch von Hand durch die Verwendung
    von Kommandozeilenbefehle wie 'Ping', 'Traceroute' oder diversen Show-Befehle durchgeführt.
    Wenn bei Konfigurationsänderungen ein Fehler gemacht wird und dieser nicht direkt durch
    einen Test gefunden wird, kann das dazu führen, dass das gesamte Netzwerk zu einem 
    zufälligen Zeitpunkt einem Ausfall erliegt.
    In der Softwareentwicklung werden seit einigen Jahren sogenannte Unit-Tests durchgeführt,
    um die Stabilität der Sowftware zu testen und zu gewährleisten.
    Diese Tests werden entweder regelmässig durch die Entwickler oder automatisiert durch
    ein Programm durchgeführt, wenn Änderungen am Programm vorgenommen werden.
    Dadurch lassen sich Fehler früh in der Entwicklung erkennen und Fehler die früh erkannt 
    werden, sind günstiger zu beheben, als Fehler die erst im laufenden Betrieb erkannt werden.

    Ziel dieser Studienarbeit ist, eine Software zu entwickeln, mit der man Unittests
    auf beliebige Netzwerke ausführen kann. 
    Ein starker Fokus wurde dabei auf die Erweiterbarkeit der Software um weitere Tests gelegt.
    Weiterhin soll darauf geachtet werden, dass die Software möglichst unabhängig von externen
    Programmen ausgeführt werden kann.

    Aus dieser Arbeit ist eine Python-Software entstanden, mit der man automatisiert Netzwerktests
    auf beliebige Netzwerke ausführen kann.
    Die Software baut auf dem Nornir-Automatisierungs-Framework auf, ein Python-Modul, welches
    eine Vielzahl von Methoden und Kommunikationsschnittstellen bietet, 
    mit denen man mit Netzwerkgeräten kommunizieren kann.
    Die Umsetzung als reines Python-Programm erlaubt es, die gesamte Software zu erweitern, 
    ohne dass neben Python eine weitere Programmiersprache erlernt werden muss.
    Neue Netzwerktests lassen sich einfach hinzufügen, ohne dass dabei ein grosser Teil der 
    Software angepasst werden muss.



    

\end{document}