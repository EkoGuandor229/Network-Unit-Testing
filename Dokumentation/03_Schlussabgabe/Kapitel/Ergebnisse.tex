\documentclass[]{subfiles}

\begin{document}
\section{Ergebnisse}
\subsection{Resultat der Arbeit}
Die Studienarbeit hat ein Python-Programm entwickelt, welches basierend auf einer Testdefinition
im YAML-Format automatische Tests gegen ein Netzwerk ausführen kann.

Kern der Software ist das Nornir-Modul für Python, ein Framework welches in Python geschrieben
ist und die Automation von Netzwerktätigkeiten ermöglicht. 
Die Auswahl von Nornir ist das Ergebnis der Evaluation möglicher Tools für die Software,
wobei schon zu Beginn der Analysephase die Verwendung von Nornir feststand, da das
Framework ein breites Spektrum von Funktionen für die Durchführung und Auswertung von 
Netzwerktests bietet.
Nornir erlaubt dem Anwender, automatisiert Konfigurationsänderungen oder -abfragen an ein 
Netzwerk zu senden.

Eine mögliche parallele Ausführung der Netzwerktests wurde in der frühen Phase des Projekts
angesprochen, konnte aber aufgrund des engen Zeitrahmens der Arbeit nicht umgesetzt werden.
Stattdessen wurde die Zeit investiert, damit die Implementierten Tests korrekt funktionieren
und die Erweiterung um neue Tests so einfach wie möglich ist.
Die Erweiterung um eine parallele Ausführung wird in die künftigen Arbeiten übernommen.

Die im Kapitel~\ref{subsec:Aufgabenstellung}-Aufgabenstellung genannten Anforderungen an 
die Tests werden vom Programm vollständig erfüllt:

\paragraph{Planbarkeit von Tests/Test-Spezifikation} 
Die Testdefinition entspricht dem Testplan/ der Test-Spezifikation.
Netzwerktests werden in der Testdefinition spezifiziert und in der Defininitionsreihenfolge,
sofern nicht über das Grafische Userinterface anders definiert, durchgeführt.

\paragraph{Dokumentation von Testresultaten} 
Die Testresultate werden in einem txt-File mit dem Datum und Zeitpunkt der Durchführung 
gespeichert. 
Somit lassen sich sämtliche vergangenen Testdurchführungen nachvollziehen.

\paragraph{Automatische Testdurchführung} 
NUTS2.0 lässt sich vollautomatisch durchführen.
Eine Orchestrierungssoftware muss dafür lediglich den Kommandozeilenbefehl
'python -m nuts -r' zu festgelegten Zeitpunkten ausführen und das 
Programm läuft danach selbstständig durch und dokumentiert die Ergebnisse.

\paragraph{Reproduzierbarkeit/Nachvollziehbarkeit der Ergebnisse} 
Wenn sich die Konfiguration im Netzwerk zwischen der Testdurchführung nicht verändert,
wird auch das Testergebnis gleich ausfallen. 
Testergebnisse werden in einem einheitlichen Format gespeichert und bei nicht bestandenen
Tests wird zusätzlich zum Testnamen noch das erwartete und das tatsächliche Ergebnis
ausgegeben, um einen Soll-Ist-Vergleich durchführen zu können.

\subsection{Programmausführung}
Das Programm kann auf einem Beliebigen Betriebssystem, welches Python installiert hat,
ausgeführt werden.
Mit dem Befehl 'python -m nuts' im Order 'Nuts2.0' wird das Programm gestartet und
lädt selbstständig die Informationen aus dem Inventar und die Testdefinitionen.
Mit dem Kommandozeilenparameter '-r' kann dabei das Grafische Userinterface übersprungen
werden, um Netzwerktests automatisiert ausführen zu lassen.
Das GUI wird verwendet, um spezifische Tests zu selektieren und die Reihenfolge der 
selektierten Tests festzulegen.
Wird das GUI übersprungen, werden alle Netzwerktests in der Reihenfolge durchgeführt,
in der sie in der Testdefinition angegeben wurden.

Die Testergebnisse werden nach der Testdurchführung in der Konsole angezeigt und 
in einem Result.txt, unter Angabe des Zeitstempels gespeichert. 
Erfolgreiche Tests werden nur mit dem Testnamen aus der Testdefinition und dem Vermerk
'PASSED' ausgegeben/gespeichert, da hauptsächlich die nicht bestandenen Netzwerktests 
relevant sind.
Nicht bestandene Tests haben zusätzlich zum Testnamen aus der Testdefinition noch das
erwartete Ergebnis und das tatsächliche Ergebnis für einen Soll-Ist-Vergleich.

\newpage

\subsection{Geräte-/Herstellerunterstützung}
Das Modul Nornir erlaubt mit seinen Plugins eine breite Unterstützung von diversen 
Herstellergeräten. 
Verschiedene Plugins haben dabei ein unterschiedliches Rückgabeformat, welches 
in den konkreten Netzwerktests in eine Normalform gebracht werden muss, damit 
NUTS2.0 einheitlich damit umgehen kann.

Nachfolgend sind die verschiedenen Plugins für die Verbindungsschnittstellen beschrieben:

\paragraph{Napalm}
Abkürzung für "Network Automation and Programmability Abstraction Layer with Multivendor support".
Napalm ist eine Pyhton Library welche verschiedene Funktionen anbietet, mit denen man 
mit Netzwerkgeräten über eine einheitliche Schnittstelle kommunizieren kann.

\paragraph{Paramiko}
Paramiko ist eine Python-Implementation des SSHv2 Protokolls für die sichere Kommunikation
zwischen verschiedenen Endgeräten.

\paragraph{Netmiko}
Netmiko ist eine Library, welche die Paramiko SSH Verbindung vereinfacht. 
Das Ziel von Netmiko ist, für verschiedene Herstellergeräte eine einheitliche Schnittstelle
zu bieten und die Kommunikation zwischen Endgeräten und Server zu vereinfachen.

\paragraph{Netconf}
Das Network Configuration Protocoll ist ein Protokoll für das Netzwerk Management.
Netconf wurde als RFC 6241 publiziert und bietet Mechanismen für die Installation,
Manipulation und das Löschen von Konfigurationen auf Netzwerkgeräten.

\newpage

\subsection{Implementierte Netzwerktests}
Die nachfolgende Auflistung beschreibt die Netzwerktests, die zum Zeitpunkt des Projektabschlusses
implementiert sind.
In erster Linie wurden Tests implementiert, die auf das gegebene Testnetzwerk ausgeführt werden
konnten.
Im Kapitel~\ref{subsec:Künftige Arbeiten}-Künftige Arbeiten werden weitere Netzwerktests beschrieben,
die in das System integriert werden können.

\begin{table}[h!]
    \begin{tabularx}{\textwidth}{lX}
        \toprule
        Testname & Testbeschreibung \\
        \midrule
        Napalm Ping & Test der einen Ping auf ein Zielgerät mit dem Napalm-Treiber durchführt.\\
        Netmiko Ping & Test der einen Ping auf ein Zielgerät mit dem Netmiko-Treiber durchführt. \\
        Napalm show Interfaces & Test der mit dem Napalm-Treiber die Interfaces des Hostgeräts anzeigt.\\
        Netmiko show Interfaces & Test der mit dem Netmiko-Treiber die Interfaces des Hostgeräts anzeigt.\\
        Netmiko Traceroute & Test der die Hops zwischen dem Hostgerät und einem Zielgerät anzeigt.\\
        Napalm show ARP Table & Test der die ARP-Tabelle des Hostgeräts mit dem Napalm-Treiber abfragt. \\
        Netmiko show ARP-Table & Test der die ARP-Tabelle des Hostgeräts mit dem Netmiko-Treiber abfragt. \\
        Napalm OSPF Neighbors & Test der die OSPF Nachbarn des Hostgeräts mit dem Napalm-Treiber abfragt. \\
        Netmiko OSPF Neighbors & Test der die OSPF Nachbarn des Hostgeräts mit dem Netmiko-Treiber abfragt. \\
        \bottomrule
    \end{tabularx}
    \caption{Implementierte Tests}
\end{table}


Man könnte nun argumentieren, dass mehrere Tests doppelt Implementiert seien, was duplizierter Code wäre.
Allerdings kann man nicht für beliebige Hostbetriebssysteme die gleichen Netzwerktests verwenden.
Zudem gibt es bestimmte Tests, z.B. Traceroute, die sich nicht mit Napalm implementieren lassen
(zumindest nicht zum Zeitpunkt der Implementierung, ein Pull-Request für die Integrierung des 
Napalm Traceroute steht offen).
Das Programm entscheidet, basierend auf dem Betriebssystem des Hostgeräts, welcher 
konkrete Test instanziiert werden soll. 
Napalm wurde dabei als der default Test für die Cisco-IOS Geräte des Testnetzwerks implementiert,
mit den Netmiko-Treiber als Fallback und für Tests, die sich mit dem Napalm-Treiber nicht umsetzen liessen.



\end{document}