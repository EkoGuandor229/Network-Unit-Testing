\documentclass[]{subfiles}

\begin{document}
\section*{Persönlicher Bericht Janik Schlatter}
Das Thema klang sehr spannend, obwohl ich mir bei der Vorstellung davon noch kein klares Bild machen konnte, 
wie so ein Projekt aussehen sollte.
Die Vorarbeit hat dabei geholfen, obwohl wir mit einer komplett anderen Technologie arbeiteten. 
Ich kenne keine 'business' Lösungen welche diese Problemzone anspricht.
Somit wahr ich gespannt wie wir dieses Problem angehen sollten.

Am Anfang gingen etliche Stunden einmal in die Einarbeit in das Thema und die Dokumentation. 
Die Einarbeit war sehr spannend und die Technologie, welche uns Urs vorgeschlagen hatte war extrem vielversprechend. 
Die Dokumentation hingegen ist nicht so mein Ding und die ersten Wochen in der Analyse und Design Phase 
waren ein wenig anstrengend.

Danach ging es entlich an die Implementation. 
Python ist für mich eine eher neuere Programmiersprache, 
jedoch haben wir in dem Fach Cloud Infrastructure schon einmal ein wenig damit gearbeitet, 
was es ein wenig einfacher machte. 
Trotzdem waren ein Paar Anlaufschwierigkeiten zu bemerken. 
Nornir ist auch ein sehr spannendes Modul und es ist extrem mächtig. 
Es hat uns bei diesem Projekt eine Menge Türen geöffnet. 
Als dann das ganze Programm langsam ins rollen kam, machte es auch richtig Spass.

Mike und ich haben bereits an einigen Projekten zusammengearbeitet und auch Computernetze 2 zusammen besucht, 
weshalb die Zusammenarbeit mit ihm auch sehr gut funktioniert.
Falls dennoch Unstimmigkeiten entstanden waren unsere Betreuer stehts zur Verfügung. 
Sie standen immer für Meetings oder Code Reviews, was uns das Projekt extrem erleichtert hat.

Alles in Allem war das Projekt sehr spannend und eine guet Erfahrung, auch wenn es uns ein Paar Nerven gekostet hat.

\end{document}