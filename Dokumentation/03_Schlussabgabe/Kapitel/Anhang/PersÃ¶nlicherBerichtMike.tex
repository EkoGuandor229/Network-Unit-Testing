\documentclass[]{subfiles}

\begin{document}
\section*{Persönlicher Bericht Mike Schmid}
    Zu Beginn der Studienarbeit war ich nicht sicher, was mich erwartet.
    Als Quereinsteiger hatte ich am Anfang des Studiums einige Schwierigkeiten,
    mir die Konzepte der Informatik, insbesondere der Softwareentwicklung anzueignen.
    Schon seit dem ersten Semester habe ich mich stark für das Thema Netzwerke und IT-Infrastrukturen
    interessiert, da ich als gelernter Elektriker bereits grundlegende Erfahrungen auf dem Gebiet hatte.
    Ich hatte bereits einige male mit der Programmiersprache Pyhton zu tun, aber ich habe noch 
    nie ein Projekt dieser Grössenordnung damit umgesetzt.

    Der Anfang der Arbeit verging relativ Reibungslos. 
    Janik Schlatter und ich haben bereits zusammen die Module Computernetze 2 und Cloud Infrastrukturen
    besucht und wir kennen unsere gegenseitigen Arbeitshaltungen aus verschiedenen gemeinsamen
    Projekten an der HSR.
    Die Zusammenarbeit mit Beat Stettler und Urs Bauman war angenehm und von gegenseitigem Respekt
    geprägt.
    Wir hatten wöchentlich ein Meeting und es wurde uns immer die Möglichkeit gegeben,
    Fragen zu stellen und um Hilfe zu bitten.
    Urs war in der Planung und Durchführung eine grosse Hilfe und egal zu welchem Thema, er konnte
    stets eine mögliche Lösung nennen oder zumindest einen Anhaltspunkt für die nächste Google-Suche
    liefern.
    
    Den grösste Frustrationsmoment hatte ich in der Meilensteinbesprechung zur Requirementsanalyse.
    Ich dachte, wir hätten die Anforderungen verstanden und haben die Arbeit dementsprechend geplant.
    Uns wurde von Kollegen, die bereits bei Herrn Stettler eine Studienarbeit durchgeführt haben,
    nahegelegt, besonders auf die Anforderungen zu achten.
    In der Besprechung wurde mir klar, dass wir die Anforderungen nicht wirklich verstanden hatten,
    obwohl ich in den vorhergehenden Meetings immer darauf geachtet habe, die Requirements 
    gewissenhaft zu ermitteln. 
    Nach diesem Meeting haben Janik und Ich kurz unsere Gehirne übertaktet und die gesamte 
    Anforderungsanalyse am selben Tag überarbeitet.
    
    Der Rest der Studienarbeit ist dann relativ ruhig verlaufen.
    Wir haben an unserem Programm gearbeitet, einmal in der Woche das Meeting durchgeführt, 
    Feedback eingeholt und weitergearbeitet.

    Mir ist vor allem der Unterschied zum Engineering-Projekt aufgefallen. Dort haben wir zu 
    viert an einem Softwareprojetkt gearbeitet und wir hatten kaum Zeit, die Software grundlegend
    fertigzustellen. 
    In der Studienarbeit habe ich zum ersten mal Iterativ an einer Software gearbeitet.
    Wenn uns aufgefallen ist, dass ein Stück Code angepasst werden muss, um eine andere Funktionalität
    zu erfüllen, haben wir diese Änderungen vorgenommen und die restiliche Software dementsprechend angepasst.

    Gegen den Abschluss der Arbeit bin ich nochmals ein wenig in Stress geraten, da die Fertigstellung
    der Sofware näherrückte und ich in der zweitletzten Semesterwoche noch WK hatte. 
    Der Zeitdruck, gleichzeitig das Programm fertigzustellen und nebenbei die Abschlussdokumentation
    zu erstellen, war zu dieser Zeit recht hoch. 

    Trotz der, nachträglich betrachtet eher geringen, Schwierigkeiten, hatte ich grosse Freude an
    der Studienarbeit und ich habe viel gelernt. Gerne werde ich weiterhin an dem Projekt 
    weiterarbeiten und ich hoffe, dass ich in der Bachelorarbeit ebenso ein Lehrreiches Projekt
    durchführen kann.
    

\end{document}