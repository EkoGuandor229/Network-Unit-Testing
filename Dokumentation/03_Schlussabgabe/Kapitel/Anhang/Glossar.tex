\documentclass[]{subfiles}

\begin{document}
\section*{Glossar}

\subsection*{Begriffserklärung}

    \begin{table}[h!]
        \begin{tabularx}{\textwidth}{lX}
            \toprule
            Begriff & Erklärung \\
            \midrule
            NUTS & Netzwork Unit Testing System \\
            SSH & Secure Shell. Übertragungsprotokoll für die verschlüsselte Kommunikation im Internet. \\
            OSPF & Open shortest path first ist ein Routing Protokoll, welches breit in der Netzwerkwelt verwendet wird. \\
            BGP & Border gateway protocol ist ein oft verwendetes Routing Protokoll. \\
            CLI & Command line interface. Abkürzung für die Kommandozeile. \\
            YAML & Menschenlesbare Serialisierungssprache. \\
            GUI & Grafisches User-interface oder auch Grafische Benutzeroberfläche. \\
            RFC & Request for comment. Standardisierungsdokument für digitale Technologien. \\
            ARP & Address resolution protocol. Protokoll für die Addressauflösung in Netzwerken. \\
            IOS & Cisco Router Betriebssystem. \\
            ISO & International standartisation organisation. \\
            IEC & International electrical commitee. \\
            XML & Extenible Markup Language. Serialisierungssprache \\
            JSON & JavaScript object notation. Serialisierungssprache \\
            OS & Operating System. Betriebssystem. \\
            SQL & Structured query language. Beschreibungssprache für Datenbankabfragen. \\
            ECTS & European Credit Transfer and Accumulation System. Berechnungssystem für Studienleistungen. \\
            IP & Internet Protokoll. \\
            ICMP & Internet Control Message Protocol. \\
            MAC & Media Access Control. \\
            \bottomrule
        \end{tabularx}
        \caption{Begriffserklärung}
    \end{table}

\clearpage
\phantomsection
\listoffigures*
\clearpage
\listoftables*

\newpage
\subsection*{Literatur- und Quellenverzeichnis}
\begin{table}[h!]
    \begin{tabularx}{\textwidth}{Xl}
        \toprule
        Author & Beschreibung \\
        Link & \\
        \midrule
        David Gee & Git for Network Engineers Blog \\
        \url{https://dave.dev/blog/2015/04/git-for-network-engineers/} & \\
        \midrule
        Learnpython.org & Diverses Referenzmaterial und Tutorials zu Python \\
        \url{https://www.learnpython.org/} & \\
        \midrule
        Pytest Dokumentation & \\
        \url{https://docs.pytest.org/en/latest/} & \\
        \midrule
        Cisco & Python Network Automation \\
        \url{https://developer.cisco.com/site/python/} & \\
        \midrule
        David Barroso & Nornir Dokumentation \\
        \url{https://nornir.readthedocs.io/en/latest/} & \\
        \midrule
        David Barroso & Napalm Dokumentation \\
        \url{https://napalm.readthedocs.io/en/latest/} & \\      
        \midrule   
        Mokhtar Ebrahim & Tkinter Python-GUI Tutorial \\
        \url{https://likegeeks.com/python-gui-examples-tkinter-tutorial/} & \\
        \midrule
        Andreas Ulrich, Sylvia Jell, Anjelika Votintseva and Andres Kull & Whitepaper zu Testdefinitionssprachen \\         
        \url{https://www.researchgate.net/publication/274634887_The_ETSI_test_description_language_TDL_and_its_application} & \\
        \midrule
        Nancy Griffeth & Whitepaper zu Netzwerktesting \\
        \url{http://comet.lehman.cuny.edu/griffeth/NetworkTesting.html} & \\
        \midrule        
        Jason C McDonald & Blog zum Thema Pyhton-Projektstruktur \\
        \url{https://dev.to/codemouse92/dead-simple-python-project-structure-and-imports-38c6} & \\
        \bottomrule
    \end{tabularx}
    \caption{Literatur- und Quellenverzeichnis}
\end{table}
\end{document}