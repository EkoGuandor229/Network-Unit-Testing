\documentclass[
	ngerman,
	toc=listof, % Abbildungsverzeichnis sowie Tabellenverzeichnis in das Inhaltsverzeichnis aufnehmen
	toc=bibliography, % Literaturverzeichnis in das Inhaltsverzeichnis aufnehmen
	footnotes=multiple, % Trennen von direkt aufeinander folgenden Fußnoten
	parskip=half, % vertikalen Abstand zwischen Absätzen verwenden anstatt horizontale Einrückung von Folgeabsätzen
	numbers=noendperiod % Den letzten Punkt nach einer Nummerierung entfernen (nach DIN 5008)
]{scrartcl}
\pdfminorversion=5 % erlaubt das Einfügen von pdf-Dateien bis Version 1.7, ohne eine Fehlermeldung zu werfen (keine Garantie für fehlerfreies Einbetten!)

% Dokumenteninformationen ----------------------------------------------------
\newcommand{\titel}{User Manual}
\newcommand{\untertitel}{Studienarbeit \semester}
\newcommand{\kompletterTitel}{\titel{} \\ \untertitel}
\newcommand{\datum}{\today}

\newcommand{\vorlagenOrdner}{../99_Vorlagen} % Falls im Unterordner ../ vorne hinzufügen

\newcommand{\betriebLogo}{\vorlagenOrdner/Bilder/logo}

% Konfiguration -------------------------------------------------------------
\newcommand{\autoren}{
    \author{
        Schmid, Mike\\
        \texttt{sgschwin@hsr.ch}
        \and
        Schlatter, Janik\\
        \texttt{jschlatt@hsr.ch}
    }
}

\newcommand{\betreuer}{
    Stettler Beat\\
    \scriptsize \texttt{\url{beat.stettler@hsr.ch}}
    \normalsize
}

\newcommand{\schmid}{
    Mike Schmid\\
    \url{mschmid@hsr.ch}
    \normalsize
}

\newcommand{\schlatter}{
    Janik Schlatter\\
    \scriptsize \url{jschlatt@hsr.ch}
    \normalsize
}

\newcommand{\autorenNamen}{
    M. Schmid, J. Schlatter
}

\newcommand{\semester}{FS-2020}
\newcommand{\betriebName}{\textsc{HSR} Hochschule für Technik Rapperswil} % Metadaten zu diesem Dokument (Autor usw.)
\input{\vorlagenOrdner/Konfiguration/Packages} % verwendete Packages
\input{\vorlagenOrdner/Konfiguration/Seitenstil_Bericht} % Definitionen zum Aussehen der Seiten
\input{\vorlagenOrdner/Konfiguration/Befehle} % eigene allgemeine Befehle, die z.B. die Arbeit mit LaTeX erleichtern

\begin{document}

% Deckblatt ------------------------------------------------------------------
\phantomsection
\thispagestyle{plain}
\pdfbookmark[1]{Deckblatt}{deckblatt}
\begin{titlepage}
    \begin{center}
        \includegraphics[scale=1.5]{\betriebLogo}\\[10ex]

        \rule{\linewidth}{0.5mm}\\[2ex]
        {\huge \bfseries  \titel }\\[2ex]
        {\LARGE \untertitel }\\[2ex]
        {\large \datum}\\
        \rule{\linewidth}{0.5mm}\\[10ex]

        \begin{minipage}[t]{0.4\textwidth}
            \begin{flushleft} 
                \large \emph{Autoren:}\\
                    \large Mike \textsc{Schmid}\\
                    \scriptsize \texttt{mike.schmid@hsr.ch}\\[1ex]
                    \large Janik \textsc{Schlatter}\\
                    \scriptsize \texttt{janik.schlatter@hsr.ch}\\[1ex]
            \end{flushleft}
            \end{minipage}
            ~
            \begin{minipage}[t]{0.4\textwidth}
            \begin{flushright} 
                \large \emph{Supervisor:} \\
                Prof. Stettler \textsc{Beat}\\
                \scriptsize \texttt{beat.stettler@hsr.ch}\\[1ex]
            \end{flushright}
        \end{minipage}\\[40ex]

        \small
        \noindent
        Dieses Werk einschließlich seiner Teile ist \textbf{urheberrechtlich geschützt}.
        Jede Verwertung außerhalb der engen Grenzen des Urheberrechtgesetzes ist ohne
        Zustimmung des Autors unzulässig und strafbar. Das gilt insbesondere für
        Vervielfältigungen, Übersetzungen, Mikroverfilmungen sowie die Einspeicherung
        und Verarbeitung in elektronischen Systemen.

    \end{center}
\end{titlepage}
\cleardoublepage

% Preface --------------------------------------------------------------------
\pagenumbering{Roman}

% Inhaltsverzeichnis
\phantomsection
\pdfbookmark[1]{Inhaltsverzeichnis}{inhalt}
\tableofcontents
\cleardoublepage

\pagenumbering{arabic}
% Jede Überschrift 1 auf neuer Seite
\let\stdsection\section
\renewcommand\section{\clearpage\stdsection}

% Inhalt ---------------------------------------------------------------------
\section{Installation}
	Es wird keine spezielle Installation benötigt. Das Repository auf GitHub muss lokal geklont werden.
	Danach muss man noch das requirements File installieren mit dem Befehl: 'pip install requirements.txt'.
	\subsection{Ausführen}
	Das Programm kan regulär in einer Programmierumgebung wie zum Beispiel PyCharm ausgeführt werden.
	Um es in einer Konsole zu starten muss zum Ordner NUTS2.0 navigiert werden. 
	Man kann es mit dem Befehl: 'python -m nuts' starten. Wenn man das GUI auslassen möchte und direkt alle Tests 
	ausführen möchte kann man mit dem Befehl: 'python -m nuts -r' starten.
	\subsubsection{Konfiguration}
	Im File 'Config.yaml' können die Pfäde aller Ordner geändert werden. 
	Zusätzlich kann noch bestimmt werden, ob das GUI Per default übersprungen werden soll.

\section{Inventar}
	Um Netzwerktests auszuführen braucht man zuerst ein Inventar mit Devices und Device Connections.
	Die Devices sind die Netzwerkgeräte wie zum Beispiel Router oder Switches.
	Die Device Connections sind die Verbindungen zwischen den Devices.

	\subsection{Devices}
		Die Definitionen der Devices sind unter Resources/Inventory/Devices/Devices.yaml:\\
		\includegraphics[scale=0.8]{\vorlagenOrdner/Bilder/Manual/Devices_yaml.png}

		Um Devices zu erfassen müssen folgende Informationen im yaml eingegeben werden:\\
		\begin{tabularx}{\textwidth}{ll}
			\toprule
			Attribut & Beschreibung \\
			\midrule
			device\_id & Eine eindeutige ID für das Device \\
			platform & Das OS welches das Device benutzt \\
			username & Der username welches das Device für das Login benutzt\\
			password & Das Passwort welches das Device für das Login benutzt\\
			hostname & Die Ip Adresse über welche das Device angesprochen werden kann\\
			\midrule
		\end{tabularx}

		Diese Informationen sollten wie folgt dargestellt werden: \\
		\includegraphics[scale=0.8]{\vorlagenOrdner/Bilder/Manual/Devices.png}

	\subsection{Device Connections}
		Die Definitionen der Device Connections sind unter Resources/Inventory/DeviceConnections/DeviceConnections.yaml:\\
		\includegraphics[scale=0.8]{\vorlagenOrdner/Bilder/Manual/DeviceConnections_yaml.png}

		Um Device Connections zu erfassen müssen folgende Informationen im yaml eingegeben werden:\\
		\begin{tabularx}{\textwidth}{ll}
			\toprule
			Attribut & Beschreibung \\
			\midrule
			device a & Die ID des ersten Devices \\
			device b & Die ID des zweiten Devices \\
			connection speed & Die Übertragungsrate der Verbindung\\
			\midrule
		\end{tabularx}

		Diese Informationen sollten wie folgt dargestellt werden: \\
		\includegraphics[scale=0.8]{\vorlagenOrdner/Bilder/Manual/DeviceConnections.png}

\section{Netzwerktests}
	Die Netzwerktests sind die Tests, welche effektiv auf dem zu testenden Netzwerk ausgeführt werden.
	Die Testdefinitionen befinden sich unter Resources/Inventory/TestDefinitions:\\
	\includegraphics[scale=0.8]{\vorlagenOrdner/Bilder/Manual/TestDefinitions_yaml.png}

	Es können in diesem Ordner beliebig viele YAML Files abgelegt werden und es werden vonn allen Files die Tests erfasst.

	Um die Tests zu erfassen müssen folgende Informationen im yaml eingegeben werden:\\
	\begin{tabularx}{\textwidth}{ll}
		\toprule
		Attribut & Beschreibung \\
		\midrule
		test\_id & Eine eindeutige ID für den Test \\
		command & Ein Command um den Test zu bestimmen \\
		test\_device & Die ID des Devices auf welchem der Test ausgeführt werden soll\\
		target & Das Ziel (Zum Beispiel eine IP im Falle eines Pings)\\
		expected\_result & Das erwartete Resultat (Zum Beispiel Success im Falle eines Pings)\\
		test\_group & Ein Gruppenname um später die Tests zu Kategorisieren\\
		\midrule
	\end{tabularx}

	Diese Informationen sollten wie folgt dargestellt werden: \\
	\includegraphics[scale=0.8]{\vorlagenOrdner/Bilder/Manual/TestDefinitions.png}
	\newpage

	\subsection{Commands}
		Folgende Commands sind bereits implementiert und können mit den jeweiligen expected\_results verwendet werden:
		\subsubsection{Ping}
			Als erwartetes Resultat kann 'Success' oder 'Failure' verwendet werden.
		\subsubsection{Show Interfaces}
			Dies benötigt kein Target also kann bei der definition einfach 'No Target' eingegeben werden.
			Als erwartetes Resultat wird ein Dictionary mit key:'Interfacename' und value:'True' oder 'False' verwendet werden.
			Dies sollte in folgender Form dargestellt werden:\\
			\includegraphics[scale=0.8]{\vorlagenOrdner/Bilder/Manual/ShowInterfaces.png}
		\subsubsection{Traceroute}
			Als erwartetes Resultat wird ein Array von IP Adressen angegeben.
			Dies sollte in folgeder Form dargestellt werden:\\
			\includegraphics[scale=0.8]{\vorlagenOrdner/Bilder/Manual/Traceroute.png}
		\subsubsection{Arp Table}
			Dies benötigt kein Target also kann bei der definition einfach 'No Target' eingegeben werden.
			Als erwartetes Resultat wird ein Array mit Dictionaries erwartet. In den Dictionaries werden folgende Informationen benötigt:
			'interface' : 'interfacename', 'mac' : 'macadresse', 'ip' : 'ipadresse'\\
			Dies sollte in folgeder Form dargestellt werden:\\
			\includegraphics[scale=0.8]{\vorlagenOrdner/Bilder/Manual/ArpTable.png}
		\subsubsection{Ospf Neighbor}
			Dies benötigt kein Target also kann bei der definition einfach 'No Target' eingegeben werden.
			Als erwartetes Resultat wird ein Array mit Dictionaries erwartet. In den Dictionaries werden folgende Informationen benötigt:
			'Neighbor-ID' : 'ipadresse', 'Priority' : 'Nummer', 'State' : 'Den Status', 'Address' : 'ipadresse', 'Interface' : 'Interfacename'\\
			Dies sollte in folgeder Form dargestellt werden:\\
			\includegraphics[scale=0.6]{\vorlagenOrdner/Bilder/Manual/OspfNeighbor.png}

\section{Durchführung}
	Nachdem dass Inventar erstellt und die Testdefinitionen erfasst wurden kann man das Programm starten.
	Falls die Option Skip-GUI aktiviert wurde werden alle Tests durchgeführt. Falls dies nicht aktiviert wurde
	öffnet sich ein GUI.

	\subsection{GUI}
		Das GUI für die Definition der Test Reihenfolge besteht aus zweit Tabs:\\
		\includegraphics[scale=0.8]{\vorlagenOrdner/Bilder/Manual/GUI1.png}

		Im ersten Tab werden alle Tests nach Gruppen sortiert angezeigt. 
		Hier kann der Benutzer auswählen welche Tests er ausführen möchte.
		Nachdem die Tests ausgewählt wurden weden mit dem Button 'Select' alle ausgewählten Tests in den zweiten Tab gezogen.

		\includegraphics[scale=0.8]{\vorlagenOrdner/Bilder/Manual/GUI2.png}

		Im zweiten Tab werden alle ausgewählten Tests angezeigt und der Benutzer kann mit den jeweiligen Buttons die Reihenfolge bestimmen.
		Nachdem der Benutzer mit der Reihenfolge zufrieden ist kann mit dem Button 'Save and Quit' das GUI beendet werden.

	\subsection{Test Resultate}
		Die Resultate der jeweiligen Durchführungen werden in der Konsole angezeigt und zusätzlich noch in einem File gespeichert.
		In der Konsole sehen die Resultate wie folgt aus: \\
		\includegraphics[scale=0.8]{\vorlagenOrdner/Bilder/Manual/ConsoleGUI.png}

		Das File befindet sich unter: 'Resources/Inventory/TestResults/results.txt'.\\
		\includegraphics[scale=0.8]{\vorlagenOrdner/Bilder/Manual/TestResults_txt.png}

		In dem File wird zuerst die Zeit des TestRuns angegeben, danach werden zuerst die Passed Tests und danach die Failed Tests angezeigt.
		Bei den Failed Tests werden zusätzlich noch das Erwartete und das Effektive Resultat angezeigt.\\
		\includegraphics[scale=0.8]{\vorlagenOrdner/Bilder/Manual/TestResults.png}

\section{Zusätzliche Tests hinzufügen}
	Falls der Benutzer eigene Tests hinzufügen möchte müssen an folgenden Orten änderungen vorgenommen werden:
	\subsection{Conctrete Tests}
		In diesem Ordner muss ein neues File angelegt werden mit dem Test, der implementiert werden möchte.
		Bei der Strucktur des Files sollte man sich an die bisherigen Files halten.\\
		\includegraphics[scale=0.8]{\vorlagenOrdner/Bilder/Manual/concretetests.png}
	\subsection{Network Test Strategy Factory}
		In dieser Klasse ist die Test Map aufgrund welcher entschieden wird, welcher Test verwendet wird für welche Test Definition.
		Hier muss in der Test Map der neue Test vermerkt werden.\\
		\includegraphics[scale=0.8]{\vorlagenOrdner/Bilder/Manual/strategyfactory.png}




\end{document}